% Conditional branch in the case where there is only one appendix (noa = 1) or
% more than one
\ifthenelse{\value{noa} = 1}
%...................then
{\chapter*{APPENDIX: THIS IS THE FIRST APPENDIX}
\addcontentsline{toc}{chapter}{APPENDIX: THIS IS THE FIRST APPENDIX}
\chaptermark{Appendix}
\markboth{Appendix}{Appendix}
\setcounter{chapter}{1}}
%...................else
{\chapter{NUMERICAL INTEGRATION IN CLASSICAL MOLECULAR DYNAMICS}}
%...................
\label{appendixA}

\section{Lagrangian and Hamiltonian Formulations}

The \emph{Lagrangian} and \emph{Hamiltonian} formulations of classical
mechanics---shown in Eqs. \ref{eqA:Lagrangian} and \ref{eqA:Hamiltonian},
respectively---offer a more convenient formalism than the more popularly known
equations derived by Newton. \cite{CorbenClassicalMechanics} While Newton's
equations apply in three-dimensional Cartesian space, they are not generally
applicable to other coordinate systems (\eg polar and spherical-polar
coordinates) that may be a more natural way to express certain problems. For
instance, polar coordinates more naturally describe the mechanics of orbiting
bodies than standard Euclidean space.

\textbf{Lagrangian Equation.} The Lagrangian function, $L = K - V$, where $K$ is
the kinetic energy and $V$ is the potential energy, satisfies the Lagrangian
equation (Eq.  \ref{eqA:Lagrangian}) for $m$ generalized coordinates ($q_m$).
The advantage of Eq. \ref{eqA:Lagrangian} is that it is derived without any
assumption of a specific coordinate system for $q_m$. Generalized velocities are
the first time-derivative of the generalized coordinates, $\dot q_m$. These
generalized velocities are used to define the kinetic energy in the familiar
form $K = 1/2 \dot q_m ^ 2$.

Another advantage to the Lagrangian formulation of classical mechanics is that
the equations are still valid when subject to constraints on the dynamics of the
system (as long as there are fewer constraints than particles).
\cite{CorbenClassicalMechanics} This property is crucial for carrying out
constrained dynamics, such as those simulations employing the commonly-used
SHAKE, \cite{Ryckaert_JComputPhys_1977_v23_p327} RATTLE, \cite{Andersen1983} or
SETTLE \cite{Miyamoto_JComputChem_1992_v13_p952} algorithms, to name a few.

\begin{equation}
   \frac d {dt} \frac {\partial L} {\partial \dot q_m} - \frac {\partial L}
         {\partial q_m} = 0
   \label{eqA:Lagrangian}
\end{equation}
$L$ in Eq. \ref{eqA:Lagrangian} is the Lagrangian function mentioned above,
$q_m$ are the generalized coordinates of every particle in the system, and $\dot
q_m$ are the set of corresponding generalized velocities. When applying Eq.
\ref{eqA:Lagrangian} to a system in the standard Cartesian coordinates without
constraints, the familiar form of Newton's equations are recovered.
\cite{CorbenClassicalMechanics}

\textbf{Hamiltonian Equation.} The Hamiltonian formulation of classical
mechanics builds on the strengths of the Lagrangian formulation and provides a
deeper insight into the physical behavior of classical systems. Unlike the
Lagrangian, the Hamiltonian is defined as the total energy of the system: $H = K
+ V$. The Lagrangian of the system, $L = K - V$, plays an important part in
Hamilton's formulation. The degrees of freedom in Hamilton's equation (Eq.
\ref{eqA:Hamiltonian}) are the generalized coordinates $q_m$ as defined in the
Lagrangian, and their conjugate momenta, $p_m$. The generalized coordinates and
momenta are said to be canonically conjugate because they obey the relationship
given in Eq. \ref{eqA:Hamiltonian}. \cite{CorbenClassicalMechanics}

\begin{align}
   q_m & = \frac {\partial H} {\partial p_m} \nonumber \\
   p_m & = - \frac {\partial H} {\partial q_m}
   \label{eqA:Hamiltonian}
\end{align}

Now that a convenient formulation of the laws of classical dynamics are known, I
will shift the discussion toward techniques by which these equations are used to
integrate these second-order differential equations in typical molecular
dynamics simulations.

\section{Numerical Integration by Finite Difference Methods}

The equations of motion are second-order differential equations with respect to
the particle coordinates, since the force is proportional to the second
time-derivative (\ie the acceleration) of those particles. Due to the typical
size and complexity of the systems and their potentials studied in computational
chemistry, MD simulations require numerical integration of the second-order
differential equations of motion. In this section, I will describe two common
approaches to iteratively integrating Eqs. \ref{eqA:Lagrangian} and
\ref{eqA:Hamiltonian}---so-called \emph{predictor-corrector} methods and the
Verlet family of integrators.

\subsection{Predictor-corrector}

The predictor-corrector integrators are based on a simple Taylor-series
expansion of the coordinates. Knowing that the velocity and acceleration are the
first- and second-time derivatives of the particle positions, respectively, the
Taylor expansions of each of these quantities are given below.
\begin{align}
   \vec{r}_p (t_0 + \delta t) & = \vec{r}(t_0) + \delta t \, \vec{v}(t_0) +
         \delta t^2 \, \frac 1 2 \vec{a}(t_0) + \delta t^3 \, \frac 1 6 \frac
         {d^3\vec{r}(t)} {dt^3} + ... \nonumber \\
   \vec{v}_p (t_0 + \delta t) & = \vec{v}(t_0) + \delta t \, a(t) + \frac 1 2 
         \delta t^2 \, \frac {d^3\vec{r}(t)} {dt^3} + ...
   \label{eqA:Predictor} \\
   \vec{a}_p (t_0 + \delta t) & = \vec{a}(t_0) + \delta t \, \frac
         {d^3\vec{r}(t)} {dt^3} + ... \nonumber
\end{align}
The subscript $p$ in these equations emphasizes that these are the
\emph{predicted} quantities of the positions, velocities, and accelerations at
time $t_0 + \delta t$ based on the known values at time $t_0$.

It is convenient to truncate the Taylor series in Eqs. \ref{eqA:Predictor} after
the acceleration term since the acceleration at time $t_0$ can be easily
calculated from the gradient of the potential energy function. Higher order
terms are difficult to compute, and contribute a significantly smaller amount as
the time step, $\delta t$, decreases. However, by truncating the Taylor
expansion we used an approximation that will introduce systematic error of our
predicted values calculated by Eqs. \ref{eqA:Predictor} compared to their
\emph{true} values. There is a way of approximating the magnitude of the
deviation of the predicted values from Eqs. \ref{eqA:Predictor}, however, that
will allow a correction to be applied to the integrated values.

As a reminder, the gradient of the potential was used to calculate the
forces---and therefore the acceleration---on each particle when making the
initial integration step from $t_0$. The acceleration may be calculated again
using the gradient of the potential at the predicted conformations:
\begin{equation}
   \bigtriangledown V\left[\vec{r}_p(t_0 + \delta t)\right] = m \vec{a}^{\prime}
   \label{eqA:ErrorCorrection}
\end{equation}
Since systematic error has been introduced by truncating the expansion in Eqs.
\ref{eqA:Predictor}, $\vec{a}^{\prime}$ from Eq. \ref{eqA:ErrorCorrection} and
$\vec{a}_p(t_0 + \delta t)$ from Eq. \ref{eqA:Predictor} will differ. The
magnitude of this difference can be used to correct the predicted values
according to Eqs. \ref{eqA:Corrector}.

\begin{align}
   \vec{r}_c(t + \delta t) & = \vec{r}_p(t + \delta t) + c_0 \Delta \vec{a}(t +
         \delta t) \nonumber \\
   \vec{v}_c(t + \delta t) & = \vec{v}_p(t + \delta t) + c_1 \Delta \vec{a}(t +
         \delta t)
   \label{eqA:Corrector} \\
   \vec{a}_c(t + \delta t) & = \vec{a}_p(t + \delta t) + c_2 \Delta \vec{a}(t +
         \delta t) \nonumber
\end{align}
where the subscripts indicate the relationship between the \emph{c}orrected and
\emph{p}redicted quantities, and the coefficients $c_0$, $c_1$, and $c_2$ are
parametrized to maximize performance, \cite{Gear1966, Gear1971} and have the
appropriate units to satisfy each equation. \cite{Allen_Tildesley} The corrector
process can be iterated until the desired level of agreement between the
predicted and corrected values is reached.

While the predictor-corrector algorithm allows long time steps to be taken by
fixing the resulting systematic error, the corrector step requires a full force
evaluation of the system at a set of coordinates, which is the most
time-consuming portion of the calculation. As a result, the corrector step is
computationally demanding, and predictor-corrector methods have been replaced by
other integration schemes in standard practice.

\subsection{Verlet Integrators}

Among the most popular types of integrators in common use today are based on the
\emph{Verlet} algorithms. The Verlet algorithm, developed in
\citeyear{Verlet_PhysRev_1967_v159_p98} by
\citeauthor{Verlet_PhysRev_1967_v159_p98}, utilizes a Taylor series expansion of
the particle coordinates about time $t_0$. The key to the Verlet approach is to
use both the forward and reverse time steps, as shown in Eqs.
\ref{eqA:PositionExpansion}. \cite{Allen_Tildesley}

\begin{align}
   \vec{r}(t_0 + \delta t) = r(t_0) + \delta t \, \vec{v}(t_0) + \frac 1 2
         \delta t^2 \, \vec{a}(t_0) + ... \nonumber \\
   \vec{r}(t_0 - \delta t) = r(t_0) - \delta t \, \vec{v}(t_0) + \frac 1 2
         \delta t^2 \, \vec{a}(t_0) - ...
   \label{eqA:PositionExpansion}
\end{align}

Combining Eqs. \ref{eqA:PositionExpansion} gives
\begin{equation}
   \vec{r}(t_0 + \delta t) = 2 \vec{r}(t_0) + \delta t^2 \, \vec{a}(t_0) -
         \vec{r}(t_0 - \delta t)
   \label{eqA:Verlet}
\end{equation}
where the velocities have been eliminated from the expression and are therefore
unnecessary when integrating the equations of motion. Furthermore, like the
velocities, the $\delta t^3$ term also cancels, so the Verlet algorithm is not
only time-reversible given its symmetry around $t_0$ but also accurate to fourth
order in the time step. The velocities are still useful, however, to compute the
total kinetic energy and related properties, such as the instantaneous
temperature. When necessary, velocities can be approximated as the average
velocity over the time period from $t_0-\delta t$ to $t_0+\delta t$.

Performing MD using the Verlet algorithm requires storing the current positions,
`old' positions at time $t_0-\delta t$, and the accelerations at time $t_0$---a
modest cost given the accuracy of the integration scheme. However, the use of
Eq. \ref{eqA:Verlet} introduces an issue of numerical precision, since
$\vec{r}(t_0)$ and $\vec{r}(t_0 - \delta t)$ are potentially large values, while
$\delta t^2 \, \vec{a}(t_0)$ is typically quite small since the time step is
small. Since real numbers can be stored only to a limited precision, accuracy is
potentially lost when a small number is added to a difference of large numbers.
\cite{Allen_Tildesley} To address this issue and improve the way in which
velocities are handled, the leap-frog and velocity Verlet methods are discussed
below.

\textbf{Velocity Verlet.} In \citeyear{Swope_JChemPhys_1982_v76_p637},
\citeauthor{Swope_JChemPhys_1982_v76_p637} developed a variant of the Verlet
algorithm that sidesteps the potential roundoff errors and naturally stores
positions, velocities, and accelerations at the same time. A Taylor series
expansion is again used to propagate the positions, but only the $t_0+\delta t$
step is used, resulting in Eq. \ref{eqA:VelVerletPositions}.
\begin{equation}
   \vec{r}(t_0 + \delta t) = \vec{r}(t_0) + \delta t \, \vec{v}(t_0) + \frac 1 2
         \delta t^2 \, \vec{a}(t_0)
   \label{eqA:VelVerletPositions}
\end{equation}
The accelerations of the particles are computed from their positions at time
$t_0 + \delta t$, and are used to compute the velocities. To increase the
accuracy of the computed velocities, the velocity integration is divided into
two half-timesteps, shown in Eqs. \ref{eqA:VelVerletVelocities}. In this case,
the accuracy to $\delta t^4$ in the positions obtained by the Verlet algorithm
is sacrificed for improved numerical precision for finite-precision computers
and a more accurate treatment of system velocities.

\begin{align}
   \vec{v}\left(t_0 + \frac 1 2 \delta t\right) = \vec{v}(t_0) + \frac 1 2
         \delta t \, \vec{a}(t_0) \nonumber \\
   \vec{v}(t_0 + \delta t) = \vec{v}\left(t_0 + \frac 1 2 \delta t\right) +
         \frac 1 2 \delta t \, \vec{a}(t_0 + \delta t) \nonumber \\
   \vec{v}(t_0 + \delta t) = \vec{v}(t) + \frac 1 2 \delta t \left[ \vec{a}(t) +
         \vec{a}(t + \delta t) \right]
   \label{eqA:VelVerletVelocities}
\end{align}

The \emph{NAB} and \emph{mdgx} programs of the AmberTools 12 program suite (and
earlier versions, where available), utilize the velocity Verlet algorithm for
dynamics.

\textbf{Leap-frog.} A common integrator used in MD simulations is the
\emph{leap-frog} method, so-called because the computed velocities `leap' over
the computed coordinates in a manner that will be explained shortly. The main
dynamics engines in the Amber 12 program suite---pmemd and sander---use the
leap-frog integrator.

While similar to the velocity Verlet approach, the leap-frog algorithm computes
positions and accelerations of particles at integral time steps, but
computes velocities at half-integral time steps according to Eqs.
\ref{eqA:LeapFrog}.

\begin{align}
   \vec{r}(t_0 + \delta t) = \vec{r}(t_0) + \delta t \, \vec{v}\left(t_0 + \frac
         1 2 \delta t \right) \nonumber \\
   \vec{v}\left(t_0 + \frac 1 2 \delta t\right) = \vec{v}\left(t_0 - \frac 1 2
         \delta t \right) + \delta t \, \vec{a}(t_0)
   \label{eqA:LeapFrog}
\end{align}

If the velocities are required at time $t_0$, they can be estimated as the
average velocities between times $t_0 - 1/2\delta t$ and $t_0 + 1/2 \delta t$,
which is significantly more accurate than the approximation in Verlet's original
algorithm. Like the velocity Verlet algorithm, leap-frog integration sacrifices
the 4th-order accuracy in integrated positions to alleviate the aforementioned
precision and velocity issues. \cite{Allen_Tildesley}
