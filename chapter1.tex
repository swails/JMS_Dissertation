\chapter{INTRODUCTION}

\section{Origins of Computational Chemistry}
The seeds of computational chemistry were sown in the mid-1800s with Ludwig
Eduard Boltzmann's formulation of \textit{statistical mechanics}. In an era when
the existence of atoms and molecules was hotly disputed within the physics
community, Boltzmann fathered a theory in which the behavior and interaction of
individual atoms or molecules on the microscopic scale could be used to
describe and predict macroscopic phenomena. Using his theorems and equations,
it became possible to reduce the problem of simulating $\sim 10^{23}$ molecules
to simulating $\sim 1$ molecule. Boltzmann used this to great effect in
describing and deriving previously-known, phenomenological equations for ideal
gases, such as the widely known equation of state, $P V = n R T$. All that
remains to provide the foundation for using molecular simulations is the proper
description of atoms and molecules on the microscopic scale.

The theories required to accurately model the behavior of individual atoms
interacting with each other and their surroundings would not be developed until
the first half of the $20^{th}$ century with the advent of \textit{quantum
mechanics}. The limits of classical mechanics became apparent when considering
the Rayleigh-Jeans formula for calculating the spectral emission of a radiating
black body. The Rayleigh-Jeans law, given by Eq. \ref{eq:RayleighJeans} and
completely derived from the laws of classical mechanics, predicts infinite
emission at high frequencies---a clear violation of the well-established law of
conservation of energy that directly contradicts experimental observations.
\begin{equation}
   B_{\nu} (T)  = \frac{2 \nu^2 k T} {c^2}
   \label{eq:RayleighJeans}
\end{equation}
To address this apparent disparity, Max Planck suggested that the error in the
classical mechanical approach was to assume a continuous emission spectrum.
Instead, Planck suggested that the emission spectra was quantized, leading to an
equation that agreed much closer with experiment. This idea of quantized energy
emissions, while developed to reconcile the mathematics of black-body radiation
with experimental measurements, would forever change our understanding of the
microscopic world.

As quantum mechanics matured, our ability to explain and predict behavior at the
atomic scale dramatically improved. In 1929, Paul Dirac proclaimed, ``The
fundamental laws necessary for the mathematical treatment of a large part of
physics and the whole of chemistry are thus completely known, and the difficulty
lies only in the fact that the application of these laws leads to equations that
are too complex to be solved.'' Even approximations designed to simplify the
equations of quantum mechanics in molecular systems resulted in computations too
complex to apply to all but the simplest systems. With the fundamental theory
necessary to describe single molecules and the machinery required to extend
that description to experimental measurements now established, computers
provided the catalyst that thrust theoretical chemistry into a prominent role in
the field.

The next sections will describe the theory of quantum mechanics and the
approximations typically employed to simplify its equations, followed by
a description of statistical mechanics.

\subsection{Quantum Mechanics}

Twenty years after Planck introduced the idea of quantized oscillators to
explain black-body radiation, Erwin Schr\"odinger introduced a wave equation
formulation of quantum mechanics (QM). \cite{Schrodinger1926} Schr\"odinger's
equation (Eq. \ref{eq:SchrodingerEquation}) bears a strong resemblance to
Hamilton's formulation of classical mechanics by employing an analogous
\textit{Hamiltonian} operator comprised of a kinetic energy term (related to the
momentum operator) and a potential energy term.
\begin{align}
   E \Psi(\vec{x}, t) & = \hat{H} \Psi(\vec{x}) \nonumber \\
   & = \left ( -\frac {\hbar ^ 2} {2 m} \bigtriangledown ^ 2 + V(\vec{x})
   \right) \Psi(\vec{x})
   \label{eq:SchrodingerEquation}
\end{align}
where $E$ is the total energy, $\hat{H}$ is the Hamiltonian operator, and
$\Psi(\vec{x}, t)$ is the wavefunction---the central object of Schr\"odinger's
equation containing all of the information and properties inherent to the
system.

Equation \ref{eq:SchrodingerEquation} is a special form of Schr\"odinger's
equation corresponding to a stationary state (\ie the potential function is
time-independent, so the energy for that state is constant). In chemistry when
we wish to calculate observable properties of a system composed of atoms, the
kinetic energy is the sum of the kinetic energies of the atomic particles in the
system, and the potential energy is calculated as the interaction of all charged
particles---protons and electrons---in the electric field they create.

The wavefunction contains all of the information about each of the particles in
the system. As the number of particles in the system increases, so too does the
complexity of wavefunction and the effort required to solve Eq.
\ref{eq:SchrodingerEquation}. Therefore, we turn to a number of approximations
developed to simplify computing a solution to Schr\"odinger's equation.

\subsubsection{Born-Oppenheimer Approximation}

The Born-Oppenheimer approximation (BOA) is almost ubiquitous in the field.
Electrons can move far more rapidly than nucleons since electrons are roughly
\mbox{1 000} times lighter. This implies that electrons can reorganize around
moving nuclei so quickly that nuclear protons are always subject to the
potential from the average electric field generated by the electrons.

Using the BOA, the wavefunction of a molecular system can be separated into two
parts: an electronic part where the nuclei are treated as fixed point charges,
and a nuclear part where the electrons are treated as a mean field.
\cite{McQuarrie_PhysChem_1997} So critical is the BOA to computational chemistry
that it appears at the heart of nearly every aspect of the field.

The next section refers to techniques in computational chemistry dealing with
the solving the electronic wavefunction.

\subsubsection{Computational Quantum Mechanics}

The main goal of most QM calculations in chemistry and molecular physics is to
determine atomic and molecular properties of the system by estimating the
electronic part of the wavefunction from the BOA. These calculations have
provided valuable assistance to experimental investigations. QM calculations
can provide reliable measurements of molecular geometries,
\cite{Jeletic_JOrganometChem_2011_v696_p3127} potential and free energy barriers
of chemical reactions, \cite{Chandrasekhar_JAmChemSoc_1985_v107_p154} ionization
energies, \cite{Watson_ChemPhysLett_2013_v555_p235} proton affinities and
gas-phase basicities \cite{Range_PhysChemChemPhys_2005_v7_p3070}, and much more.
\cite{Hehre_Ab_initio_MO_Theory_Book_1986}

These calculations are becoming routine as more and more experimental studies
employ some form of calculation to help interpret results or strengthen
conclusions. Despite all their successes and the rapid increase of computational
power over recent years, however, the computational demands of QM methods remain
prohibitively high for systems with more than 100 -- 200 atoms. Furthermore for
researchers interested in these large systems, calculations on a single
arrangement of atomic nuclei becomes increasingly insufficient to quantify the
behavior of those systems.

For such applications, we turn our attention back to statistical mechanics with
the aim of ultimately applying those principles to molecular mechanical
simulations of large molecules that often contain thousands---even hundreds of
thousands---of atoms.

\subsection{Statistical Mechanics}

Armed with very accurate models describing how atoms and molecules behave and
interact at a molecular level, we are ready to apply the principles of
statistical mechanics to predict macroscopically observable phenomena. To begin,
we will start with the idea of an \emph{ensemble}.

Substances that we see are composed of billions of billions of individual atoms.
The number of atoms contained in things we interact with every day is so large
that chemists invented the \emph{mole} in order to work with more manageable
numbers. Similar in concept to the dozen, a mole is a collection of $6.022
\times 10 ^ {23}$ `things'---it is the number of $^{12}C$ atoms in exactly 12
grams of carbon.
