\chapter{INTRODUCTION}

\section{Origins of Computational Chemistry}
The seeds of computational chemistry were sown in the mid-1800s with Ludwig
Eduard Boltzmann's formulation of \textit{statistical mechanics}. In an era when
the existence of atoms and molecules was hotly disputed within the physics
community, Boltzmann fathered a theory in which the behavior and interaction of
individual atoms or molecules on the microscopic scale could be used to
describe and predict macroscopic phenomena. Using his theorems and equations,
it became possible to reduce the problem of simulating $\sim 10^{23}$ molecules
to simulating $\sim ~1$ molecule. Boltzmann used this to great effect in
describing and deriving previously-known, phenomenological equations for ideal
gases, such as the widely known equation of state, $P V = n R T$. All that
remains is the proper description of atoms and molecules on the microscopic
scale.

The theories required to accurately model the behavior of individual atoms
interacting with each other and their surroundings would not be developed until
the first half of the $20^{th}$ century with the advent of \textit{quantum
mechanics}. The limits of classical mechanics became apparent when considering
the Rayleigh-Jeans formula for calculating the spectral emission of a radiating
black body. The Rayleigh-Jeans law, given by equation \ref{eq:RayleighJeans} and
completely derived from the laws of classical mechanics, predicts infinite
emission at high frequencies---a clear violation of the well-established law of
conservation of energy that directly contradicts well-established experimental
observations.
\begin{equation}
   B_{\nu} (T)  = \frac{2 \nu^2 k T} {c^2}
   \label{eq:RayleighJeans}
\end{equation}
To address this apparent disparity, Max Planck suggested that the error in the
classical mechanical approach was to assume a continuous emission spectrum.
Instead, Planck suggested that the emission spectra was quantized, leading to an
equation that agreed much closer with experiment. This idea of quantized energy
emissions, while developed to reconcile the mathematics of black-body radiation
with experimental measurements, would forever change our understanding of the
microscopic world.

As quantum mechanics matured, our ability to explain and predict behavior at the
atomic scale dramatically improved. In 1929, Paul Dirac proclaimed, ``The
fundamental laws necessary for the mathematical treatment of a large part of
physics and the whole of chemistry are thus completely known, and the difficulty
lies only in the fact that the application of these laws leads to equations that
are too complex to be solved.'' Even approximations designed to simplify the
equations of quantum mechanics in molecular systems resulted in computations too
complex to apply to all but the simplest systems. With the fundamental theory
necessary to describe single molecules and the machinery required to extend
that description to experimental measurements now established, computers
provided the catalyst that thrust theoretical chemistry into a prominent role in
the field.

The next sections will describe the theory of quantum mechanics and the
approximations typically employed to simplify its equations followed by
statistical mechanics.

\section{Quantum Mechanics}

