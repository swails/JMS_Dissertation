\chapter{REMD: GPU ACCELERATION AND EXCHANGES IN MULTIPLE DIMENSIONS}
\label{ch5}

This chapter contains a description of my work implementing replica exchange
molecular dynamics (REMD) in the \emph{pmemd} program of the Amber program
suite. \cite{AMBER12}. The first sections describe the general theory of the
state exchanges supported by Amber, followed by details of their implementation.
I then describe my design of multiple-dimension REMD in Amber and provide some
sample studies performed using the code.

\section{Temperature REMD}

The most common variant of REMD simulations involves assigning replicas with
different temperatures (T-REMD). \cite{Sugita_ChemPhysLett_1999_v314_p141} In
typical T-REMD simulations, the Monte Carlo-based replica exchange attempts
occur between a pair of replicas at different temperatures. The exchange success
probability---calculated in a way that satisfies detailed balance to preserve
valid thermodynamics---is solved for the proposed change of two replicas
swapping temperatures, as shown in Eq. \ref{eq5:TExchProb}.  When $2N$ replicas
are present, $N$ independent exchange attempts can be made simultaneously
between different pairs of replicas. If no replica is involved in multiple
exchange attempts, these moves can be evaluated independently. While this may
not be the most efficient way to perform replica exchange attempts, it is the
most common approach due to its simplicity and efficiency.

To calculate the exchange probability in T-REMD exchange attempts, we start with
the detailed balance equation (Eq. \ref{eq1:DetailedBalance}) in which replicas
$m$ and $n$ have temperatures $T_m$ and $T_n$, respectively in our initial state
$i$. The temperatures swap in our proposed state such that replicas $m$ and $n$
have temperatures $T_n$ and $T_m$, respectively. Because the potential energy
function of each replica is the same---only the temperature differs between
replicas---the probability of each replica having each temperature is directly
proportional to the Boltzmann factor (in the canonical ensemble). The derivation
of the exchange probability equation in T-REMD simulations is shown in Eq.
\ref{eq5:TExchProb}.

\begin{align}
   P_{i} \pi_{i \rightarrow j} & = P_{j} \pi_{j \rightarrow i} \nonumber \\
   \frac {\exp \left[ -\beta_m E_m \right] \exp \left[ -\beta_n E_n \right]}
         {Q_m Q_n} \pi_{i \rightarrow j} & = \frac {\exp \left[ -\beta_n E_m
         \right] \exp \left[ -\beta_m E_n \right]} {Q_n Q_m} \pi_{j \rightarrow
         i} \nonumber \\
%  \exp \left[ -\beta_m E_m - \beta_n E_n \right] \pi_{i \rightarrow j} & =
%        \exp \left[ -\beta_n E_m - \beta_m E_n \right] \pi_{j \rightarrow i}
%        \nonumber \\
%  \frac {\pi_{i \rightarrow j}} {\pi_{j \rightarrow i}} & = \frac {\exp \left[
%        -\beta_n E_m - \beta_m E_n \right]} {\exp \left[ -\beta_m E_m - \beta_n
%        E_n \right]} \nonumber \\
%  \frac {\pi_{i \rightarrow j}} {\pi_{j \rightarrow i}} & = \exp \left[
%        -\beta_n E_m - \beta_m E_n + \beta_m E_m + \beta_n E_n \right]
%        \nonumber \\
   \frac {\pi_{i \rightarrow j}} {\pi_{j \rightarrow i}} & = \min \left \lbrace
         1, \exp \left[ (\beta_n - \beta_m) (E_n - E_m) \right] \right \rbrace
   \label{eq5:TExchProb}
\end{align}
where $\beta_m$ is $1/k_BT_m$ for replica $m$ and $E_m$ is the potential energy
of the structure in replica $m$.

Because the temperature of the system uniquely defines its kinetic energy, the
potential energy can be used in lieu of the total energy in Eq.
\ref{eq5:TExchProb} as long as the total temperature remains consistent after
the exchange attempt completes. Therefore, the momenta of replica $m$ are
typically scaled by $\sqrt{T_n/T_m}$ after successfully exchanging with replica
$n$. \cite{Sugita_ChemPhysLett_1999_v314_p141} By scaling the velocities in this
way, snapshots following a successful exchange attempt are immediately
`equilibrated' members of the new temperature's ensemble, thereby eliminating
the need to relax the structure to its `new' temperature. This allows REMD
simulations to be carried out more efficiently by permitting exchange attempts
very frequently. \cite{Sindhikara2008, Sindhikara2010}

An important consideration for T-REMD simulations is how many temperature
replicas you should use as well as what temperatures those replicas should have.
As the temperature of a system increases, the number of low-energy structures
that are sampled during the simulation decreases. In fact, at infinite
temperatures, MD is effectively equivalent to random sampling, whose
consequences are illustrated in Fig. \ref{fig1:EthanMC}. The temperature ladder
(\ie the selection of temperatures at which to run each replica) should be
chosen so as to optimize the simulation efficiency. If the temperature
difference between adjacent replicas is too great, then the average potential
energy difference between adjacent replicas will be large and the exchange
probability in Eq. \ref{eq5:TExchProb} will be very small. As a result, the low
temperature ensembles will not benefit from the enhanced sampling achievable at
the higher temperatures. On the other hand, if the temperature difference
between adjacent replicas is too small, then computational effort will be wasted
by simulating unnecessary replicas that do not enhance sampling from the
generalized ensemble.

By analyzing Eq. \ref{eq5:TExchProb}, it becomes clear 
