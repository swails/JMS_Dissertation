\chapter{FLEXIBLE TOOLS FOR AMBER SIMULATIONS}
\label{ch6}

In this chapter, I will describe the motivation behind creating several tools
to aid users in carrying out biomolecular simulations with the Amber programming
package as well as some details regarding their functionality and
implementation. During the course of my graduate studies, I wrote several tools
to aid in my work---some of which I polished and released with the Amber suite
of programs. The three I will describe in this chapter are \emph{MMPBSA.py},
\cite{MMPBSApy} \emph{ParmEd}, and \emph{update\_amber}.

\section{MMPBSA.py}

Portions of this section are reprinted with permission from
\citeauthor{MMPBSApy}, ``MMPBSA.py: An Efficient Program for End-State Free
Energy Calculations,'' \emph{J. Chem. Theory Comput.}, \textbf{8} (9), pp
3314--3321. \cite{MMPBSApy}

\subsection{Motivation}

End-state free energy methods---briefly described in Sec.
\ref{sec1:EndState}---are popular methods for computing binding free energies
for protein-ligand binding, \cite{Wang2001, Kuhn2005, Weis2006, Genheden2009,
Wang2001a} protein-protein binding, \cite{Gohlke2003, Gohlke2004, Bradshaw2010,
Wang2001a} nucleic acid binding, \cite{Gouda2002, Wang2001a} and relative
conformational stabilities. \cite{Combelles2008, Brice2011} There has been
significant effort applied to improving the approximations used in end-state
methods, and in some cases it has even approached predictive accuracy.
\cite{Genheden2009, Mikulskis2012}

By 2008, there was a set of perl scripts written to perform MM-PBSA and MM-GBSA
calculations that were written in 2002 for release with Amber 7 and had not been
changed since 2003.
