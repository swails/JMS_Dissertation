% Abstract of the dissertation
\begin{abstract}
Solution pH has profound effects on the structure, function, and activity of
many complex biomolecules that catalyzing the chemical reactions responsible for
sustaining life. Even focusing on the Human body, the various physiological
environments span a wide pH range---as low as 1 in the stomach to values as high
as 8.1 in pancreatic secretions. Small changes from the normal pH of a
biomolecule's environment can be catastrophically disruptive to its activity.
For example, a change in pH of as little as $\pm$0.1 pH units in the Human
bloodstream is enough to cause life-threatening alkalosis or acidosis.

Due to the importance of pH in biology and the profound effect it can have on
biomolecules, it is important to incorporate pH effects in computational models
designed to treat these biomolecules. The solution pH controls protonation state
equilibria of specific functional groups prevalent in biomolecules, such as
carboxylates, amines, and imidazoles.  These protonation states in turn affect
the charge distribution in the biomolecule which can have a significant impact
on both its 3-dimensional structure as well as interactions with the surrounding
environment.  In many cases, pH can also impact whether or not a proton donor or
acceptor will be available for catalysis during the course of the biocatalytic
mechanism.

The aim of my work is to develop accurate, efficient computational models to
probe the pH-dependent behavior of proteins and nucleic acids. The models must
be carefully designed to obey the laws of thermodynamics under the constraint of
an externally applied pH. Only then can the results be directly compared to
experimental measurements.

In this dissertation, I present my work on the development of pH-based models
for biomolecules and other work performed in the area of molecular modelling.
In the first chapter, I introduce the fundamental concepts computational
biomolecular modeling that lays the foundation for the presented work.  This is
followed by chapters on free energy and sampling, constant pH simulations,
replica exchange, and some useful tools I developed to aid in conducting
computational research with the Amber simulation package.
\end{abstract}
