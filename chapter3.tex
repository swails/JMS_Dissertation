\chapter{CONSTANT pH REPLICA EXCHANGE MOLECULAR DYNAMICS}

In this chapter, I will discuss my work with REMD simulations in which the state
parameter---or the property exchanged between replicas---is the solution pH.

\section{Constant pH and pK\sub{a} Calculations}

Solution pH is often critical to the proper functioning of biological catalysts.
\cite{Cornish-Bowden1969,White1959}  The pH environment of biological systems
influences the ionization equilibria present in the system, thereby affecting
the protonation state of various \emph{titratable residues} in the system.  A
titratable residue is any residue that has a pK\sub{a} value within 1 or 2 units
of the biological pH range (which is roughly 2 -- 9).  The protonation states of
these residues can have a profound effect on the stability of the system, the
system's interactions with its surroundings, and any catalytic mechanism that
relies on a specific set of protonation states to carry out general acid-base
catalysis or nucleophilic attack. \cite{Tanford1957}

Simulations aimed at modeling proteins or nucleic acids must have some method
for assigning protonation states for each titratable residue.  Because bond
breaking and bond formation are impossible in classical force fields, each
residue is typically assigned one protonation state and the entire simulation is
run using this set of states.  This approach has two drawbacks.  First, the
choice of protonation state is often based on the behavior of each titratable
residue when free in solution.  This may not be a valid assumption, however,
because the protein or nucleic acid environment can modulate a residue's
protonation state equilibrium.  Second, a single protonation state may not
accurately represent the true ensemble of states at the desired pH.  If the pH
is close in magnitude to the pK\sub{a} of a given residue, or if the system
populates conformations in which the dominant protonation state changes, then
the true ensemble is represented by conformations with different protonation
states.

The first drawback can be addressed by using tools such as \emph{PROPKA}
\cite{Olsson2011} and \emph{H++}, \cite{Myers2006} which provide a means to
assign protonation states to titratable residues by calculating the pK\sub{a} of
the starting structure.  However, this does not address the possibility  that
multiple protonation states may be necessary to build the desired ensemble.

While it may seem that both drawbacks can be addressed by simply running
simulations with every possible set of protonation states, this approach quickly
becomes unwieldy.  Given $N$ titratable residues, there are $2^N$ distinct
protonation states assuming each residue is either protonated or deprotonated.
With only 10 titratable residues this amounts to 1024 distinct simulations!
While most of these states may not be found in the given ensemble, there is no
way to know which ones to exclude \emph{a priori}.  It is important, then, to
develop a method capable of directly probing protonation state equilibria in
biological molecules.

In order to probe protonation state equilibria in a thermodynamically meaningful
way, simulations must be run at constant pH.  The first approaches for constant
pH simulations used \emph{continuum electrostatics} methods to calculate the
perturbing effect of the system environment on protonation state equilibria
using an implicit solvent model (e.g., the Poisson-Boltzmann equation) on a
single structure.  \cite{Bashford1990,Bashford1992,Antosiewicz1994}  These
methods, while sometimes useful for calculating pK\sub{a} values in biological
systems, assume that the full protonation state equilibria can be characterized
with a single structure.  In particular, using a single structure neglects the
response of the system relaxing to accommodate the new protonation state.  While
this has been addressed to some degree by treating the protein interior with a
large dielectric constant, \cite{Antosiewicz1994} this approach assumes an
unphysical homogeneity in the system's dielectric response to protonation state
changes.

A more sophisticated approach to incorporating the system response involves
simultaneous sampling of both protonation states and side-chain rotamers.
\cite{Song2009}  This approach dramatically improves pK\sub{a} prediction with
respect to experiment, but may be insufficient for systems with large scale
conformational changes that cannot be attributed only to side chain mobility.

To capture the coupled nature of conformational flexibility with protonation
state sampling, several constant pH molecular dynamics (CpHMD) methods have been
proposed.  \cite{Baptista1997, Baptista2002, Burgi2002, Lee2004, Borjesson2004,
Mongan2004, Khandogin2005} These methods have proven to be powerful tools for
pK\sub{a} calculation and prediction, although there is still room for
improvement. \cite{Alexov2011}  For systems in which some titratable residues
experience large pK\sub{a} shifts, predicted pK\sub{a} values are often in error
by more than 1 pH unit even in the studies that reproduce experimental values
the closest. \cite{Alexov2011}  This is usually a direct result of insufficient
sampling of protonation and conformational states or a limitation of the
underlying model.  \citeauthor{Machuqueiro2011} have shown that correcting some
of the limitations of the underlying model, such as improving the definition of
the reference compound (whose role is described below in the \emph{Theory}
section) and improving the underlying force field improves results.
\cite{Machuqueiro2011} Other work has coupled enhanced sampling techniques, such
as accelerated molecular dynamics \cite{Hamelberg2004}, with CpHMD to show that
improved conformational sampling also improves predicted pK\sub{a}s with respect
to experiment. \cite{Williams2010} \citeauthor{Webb2011} recently published a
systematic study showing that the errors inherent to experimental measurements
are often larger than those reported, which has important implications for the
accuracy of theoretical predictions. \cite{Webb2011}

\emph{Replica exchange molecular dynamics} (REMD) is a family of extended
ensemble techniques that have been shown to dramatically improve sampling.
\cite{Sugita1999, Pitera2003, Chodera2011, Nadler2008, Meng2010, Meng2011a} In
REMD simulations, a series of independent replicas (single MD trajectories of a
system) periodically attempt to exchange information, such as temperature
\cite{Sugita1999,Pitera2003} and, more recently, pH \cite{Itoh2011,Wallace2011}
to sample from an expanded ensemble covering multiple states.

In this study, we implemented, in the \emph{sander} module of the AMBER
\cite{AMBER12} software package, the pH-REMD method described by
\citeauthor{Itoh2011} \cite{Itoh2011}.  We show how this method significantly
improves sampling compared to CpHMD in hen egg-white lysozyme (HEWL), a system
commonly used as a benchmark for pK\sub{a} calculations.  Titration curves
generated using pH-REMD contain significantly less noise and converge more
rapidly than CpHMD, suggesting pH-REMD is a powerful tool for carrying out
pK\sub{a} predictions.

Our group has previously shown that temperature REMD simulations converge
significantly faster with increasing exchange attempt frequency (EAF).
\cite{Sindhikara2008,Sindhikara2010}  Here, we show that increasing the EAF in
pH-REMD simulations causes pH-dependent observable properties to converge faster
as well.

In the next sections, I will describe the foundation of the constant pH method
developed by \citeauthor{Mongan2004} \cite{Mongan2004} and the corresponding
pH-REMD method. \cite{Itoh2011,Wallace2011} I will then describe the details of
my study on HEWL followed by the results and conclusions drawn from that study.

\section{Theory}

\subsection{CpHMD}

We used the constant pH molecular dynamics (CpHMD) method developed by
\citeauthor{Mongan2004} \cite{Mongan2004} that employs Monte Carlo transitions
between discrete protonation states at periodic intervals during a MD simulation
to probe protonation state equilibria.  In this CpHMD implementation, both the
dynamics and the MC protonation state sampling are performed in Generalized Born
implicit solvent.  After a predetermined number of steps, the MD is halted and a
protonation state change is attempted by evaluating the energetic cost of that
proposed change, calculated according to \cite{Mongan2004}
\begin{equation}
   \Delta G = k _ B T \left( pH - pK _ {a,ref} \right) \ln 10 + \Delta G _
         {elec} - \Delta G _ {elec,ref}
   \label{eq3:ProtChange}
\end{equation}
\ref{eq3:ProtChange} represents a free energy change of protonating or
deprotonating a titratable residue embedded in a biological system with respect
to a predefined reference compound.  The reference compound is a monomer of the
titratable residue capped with small, neutral functional groups.  In
\ref{eq3:ProtChange}, $\Delta G _ {elec}$ is calculated by taking the difference
of the electrostatic energy between the proposed and existing protonation
states. \cite{Mongan2004}

Directly calculating the free energy change associated with protonation or
deprotonation is difficult because evaluating the energetic cost of desolvating
a free proton and making and breaking chemical bonds is impossible in a
classical mechanical framework.  Therefore, we calculate the free energy cost of
this protonation state change by comparing the free energy of the protonation
state change to $\Delta G _ {elec,ref}$ in \ref{eq3:ProtChange}, a precomputed
free energy for the reference compound that is adjusted to reproduce
experimental pK\sub{a} values.  \ref{eq3:ProtChange}, then, represents a shift in
the pK\sub{a} of a titratable residue in a biological system from its value free
in solution. The reference compound pK\sub{a} values used in the Amber CpHMD
implementation \cite{Mongan2004} are shown in \ref{tbl3:refpkas}.

\begin{table}
  \caption{Reference pK\sub{a} values for the acidic residues treated in this
           study. Values are the same as those used in the original Amber CpHMD
           implementation.\cite{Mongan2004}}
  \label{tbl3:refpkas}
  \begin{tabular}{lc}
    \hline
    Residue & Reference pK\sub{a} \\
    \hline
    Aspartate & 4.0 \\
    Glutamate & 4.4 \\
    Histidine (H$^\delta$) & 7.1 \\
    Histidine (H$^\epsilon$) & 6.5 \\
    \hline
  \end{tabular}
\end{table}

Running a CpHMD simulation, we obtain an ensemble consisting of multiple
protonation states properly weighted for the semi-grand canonical ensemble, the
thermodynamic ensemble corresponding to constant temperature, volume (or
pressure) and chemical potential of hydronium (i.e., constant pH).
\cite{Baptista2002}  Because the simulation is assumed to be ergodic, the
deprotonation fraction can be calculated by simply counting the fraction of
ensemble members in which the residue is deprotonated. Multiple CpHMD
simulations must be run with a range of pHs to calculate pK\sub{a} values for
titratable residues in biological systems by fitting a titration curve to the
data.

Running a simulation with an expanded ensemble so each CpHMD simulation is in
equilibrium with simulations at different pHs can further enhance sampling from
the desired semi-grand canonical ensemble.  For this, we turn to the pH-REMD
method.

\subsection{pH-REMD}

Replica exchange simulations at constant pH (pH-REMD) is a variant of replica
exchange in which each replica is simulated at a separate pH.  The full pH-REMD
simulation represents an expanded ensemble in which each replica samples
conformations with a fixed pH and samples different pH values at a fixed
conformation.

In this study, we implemented the pH-REMD method introduced by
\citeauthor{Itoh2011} \cite{Itoh2011} in the \emph{sander} module of Amber.
\cite{AMBER12}  In pH-REMD, adjacent replicas in the pH ladder swap pH with the
Monte Carlo exchange probability
\begin{equation}
   P _ {i \rightarrow j} = \min \left \lbrace 1, \exp \left[ \ln 10 \left( N _ i
            - N _ j \right) \left( pH_i - pH_j \right) \right] \right \rbrace
   \label{eq3:ExchSucc}
\end{equation}
for replicas $i$ and $j$ where $N_i$ is the number of titratable protons present
in replica \emph{i} and $pH_i$ is the pH of replica $i$ prior to the exchange
attempt.

Our group recently developed a different pH-REMD method in which replica
exchanges are attempted via Hamiltonian exchange where only atomic coordinates
are swapped. \cite{SabriDashti2012} In contrast, the currently proposed method
only swaps the solution pH between replicas.  For large systems with more than 3
-- 5 titratable residues, the proposed method of swapping solution pH between
replicas achieves more efficient replica exchanges than the variant employing
Hamiltonian exchange.  For HEWL, specifically, the Hamiltonian REMD variant
experienced an exchange attempt success rate of <0.01\%, which is effectively
indistinguishable from CpHMD simulations.

\section{Methods}

\subsection{Starting Structure}

We chose to study hen egg white lysozyme because it is well-characterized both
experimentally \cite{Takahashi1992, Bartik1994, Webb2011} and computationally.
\cite{Mongan2004,Demchuk1996,Wallace2011} We chose the structure from the
protein data bank (PDB) with the code 1AKI \cite{Artymiuk1982} because it was
the focus of Mongan's original study. \cite{Mongan2004}

The topology file was prepared in the \emph{tleap} module of AmberTools 12 using
the Amber \emph{ff10} force field, which is equivalent to \emph{ff99SB}
\cite{Hornak2006} for proteins.  Crystallographic water molecules were removed
from the starting structures, and \emph{tleap} added all hydrogen atoms.
Finally, the \emph{mbondi2} intrinsic radii for implicit solvent calculations
were selected in \emph{tleap} to be consistent with the initial implementation
of CpHMD. \cite{Mongan2004}

\subsection{Molecular Dynamics}

To be consistent with the original implementation, the Generalized Born model
described by \citeauthor{Onufriev2004} \cite{Onufriev2004} (corresponding to the
input parameter \emph{igb=2} for Amber programs) was used with the salt
concentration, modeled as a Debye screening parameter, was set to 0.1 M in every
simulation.  \cite{Mongan2004}  Due to the long-range nature of the
electrostatic forces, we always used an infinite cutoff for non-bonded
interactions.

Each starting structure was minimized using 50 steps of steepest descent
followed by 950 steps of conjugate gradient with 10 kcal mol\super{-1}
\AA\super{-2}  restraints on the backbone atoms to relieve bad contacts.  Then,
the minimized structure was heated by varying the target temperature linearly
from 10 K to 300 K for 667 ps, keeping weak restraints---1 kcal mol\super{-1}
\AA\super{-2}--on the backbone.  We used the Langevin thermostat with a
collision frequency of 5 ps\super{-1} to control the temperature.  These
simulations were performed using the \emph{pmemd} module of the Amber 12 program
suite. \cite{AMBER12}

After heating, each structure was further run at 300 K for 1 ns with 0.1 kcal
mol\super{-1} \AA\super{-2} restraints on the backbone.  Each titratable
carboxylate was deprotonated and the histidine was protonated, and no
protonation state changes were attempted during the simulation.  Next, the
resulting structure was used to start 16 ns of CpHMD at pH values spanning 2 to
7 with an interval of 0.5.  Only the 10 acidic residues---the aspartates,
glutamates, and histidines---were titrated because HEWL is catalytically active
at low pH \cite{Vocadlo2001} and all 1AKI was solved in these conditions.  We
used a 2 fs time step and attempted protonation state changes every 5 steps.
The Langevin thermostat with a collision frequency of 10 ps\super{-1} was used
to control the temperature, and simulations were begun with a different random
seed to avoid synchronization artifacts. \cite{Sindhikara2009}  We used the
\emph{sander} module of Amber 12 for each of these simulations.

