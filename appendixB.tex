\chapter{AMBER Parameter-Topology File Format}%
\label{appendixB}

This appendix details the Parameter-Topology file format used extensively by the
AMBER software suite for biomolecular simulation and analysis, referred to as
the \emph{prmtop} file for short. The format specification of the AMBER topology
file was written initially over a decade ago and posted on
http://ambermd.org/formats.html. I have recently expanded that document to
account for the drastic change to the file format that occurred with the 2004
release of Amber 7. The pre-Amber 7 format (\emph{old format}) is described more
briefly afterwards, although each section provided in the original format
contains exactly the same information as the newer version.

This appendix also details the format changes and additions introduced by
\emph{chamber}---the program that translates a CHARMM parameter file (PSF) into
a topology file that can be used with the \emph{sander} and \emph{pmemd}
programs in AMBER.

This appendix draws from the information on http://ambermd.org/formats.html that
was added by both me and others, as well as the experience I gleaned while
writing the ParmEd program and working with the various codes in AMBER.

As a warning, the prmtop file is a result of bookkeeping that becomes
increasingly complex as the system size increases. Therefore, hand-editing the
topology file for all but the smallest systems is discouraged---a program or
script should be written to automate the procedure.

\section{Layout}

The first line of the Amber topology file is the version string. An example is
shown below in which {\tt XX} is replaced by the actual date and time.
\begin{verbatim}
%VERSION  VERSION_STAMP = V0001.000  DATE = XX/XX/XX  XX:XX:XX
\end{verbatim}

The topology format is divided into several sections in a way that is designed
to be parsed easily using simple Fortran code. A consequence of this is that it
is difficult for parsers written in other languages (\eg C, C++, Python, etc.)
to strictly adhere to the standard. These parsers should try, however, to
support as much of the standard as possible.

\begin{verbatim}
%FLAG SECTION
%COMMENT an arbitrary number of optional comments may be put here
%FORMAT(<FORTRAN FORMAT>)
... data formatted according to <FORTRAN FORMAT>
\end{verbatim}

All names (\eg atom names, atom type names, and residue names) are limited to
4 characters and are printed in fields of width \emph{exactly} 4 characters
wide, left-justified. This means that names may not be space-delimited if any of
the names have 4 characters.

\paragraph{Requirements for prmtop parsers}

Parsers, regardless of the language they are written in, should conform to a
list of attributes to maximize the likelihood that they are parsed correctly.

\begin{itemize}
   \item Parsers should expect that some 4-character fields (\eg atom or residue
         names) may have some names that have 4 characters and therefore may not
         be whitespace-delimited.
   \item Parsers should not expect {\tt SECTION}s in the prmtop to be in any
         particular order.
   \item Parsers should not expect or require {\tt \%COMMENT} lines to exist,
         but should properly parse the file if any number of {\tt \%COMMENT}
         lines appear as indicated above
   \item The topology file may be assumed to have been generated `correctly' by
         \emph{tleap} or some other credible source. No graceful error checking
         is required.
\end{itemize}

\paragraph{Requirements for modifying {\tt SECTION}s}

To minimize the impact of prmtop changes to existing, third-party parsers, the
following conventions should be followed.

\begin{itemize}
   \item Any new {\tt SECTION} should be added to the end of the topology file to
         avoid conflicts with order-dependent parsers.
   \item The {\tt <FORTRAN FORMAT>} should be as simple as possible (and avoid
         adding new formats) to maintain simplicity for non-Fortran parsers.
   \item Avoid modifying if possible. Consider if this section truly belongs in
         the prmtop.
\end{itemize}

\section{List of {\tt SECTION}s}

\subsection*{TITLE}

This section contains the title of the topology file on one line (up to 80
characters). While the title serves a primarily cosmetic purpose, this section
must be present.

\noindent {\tt \%FORMAT(20a4)}

\subsection*{POINTERS}

This section contains the information about how many parameters are present in
all of the sections. There are 31 or 32 integer pointers (NCOPY may not be
present). The format and names of all of the pointers are listed below, followed
by a description of each pointer.

\vspace{8pt}
\hline
\vspace{4pt}

\begin{verbatim}
%FLAG POINTERS
%FORMAT(10I8)
NATOM  NTYPES NBONH  MBONA  NTHETH MTHETA NPHIH  MPHIA  NHPARM NPARM  
NNB    NRES   NBONA  NTHETA NPHIA  NUMBND NUMANG NPTRA  NATYP  NPHB   
IFPERT NBPER  NGPER  NDPER  MBPER  MGPER  MDPER  IFBOX  NMXRS  IFCAP  
NUMEXTRA NCOPY 
\end{verbatim}

\vspace{4pt}
\hline
\vspace{8pt}

\begin{description}
   \item[NATOM] Number of atoms
   \item[NTYPES] Number of distinct Lennard-Jones atom types
   \item[NBONH] Number of bonds containing Hydrogen
   \item[MBONA] Number of bonds not containing Hydrogen
   \item[NTHETH] Number of angles containing Hydrogen
   \item[MTHETA] Number of angles not containing Hydrogen
   \item[NPHIH] Number of torsions containing Hydrogen
   \item[MPHIA] Number of torsions not containing Hydrogen
   \item[NHPARM] Not currently used for anything
   \item[NPARM] Used to determine if this is a LES-compatible prmtop
   \item[NNB] Number of excluded atoms (length of total exclusion list)
   \item[NRES] Number of residues
   \item[NBONA] MBONA + number of constraint bonds \footnote{AMBER codes no
         longer support constraints in the topology
         file.}\addtocounter{footnote}{-1}\addtocounter{Hfootnote}{-1}
   \item[NTHETA] MTHETA + number of constraint angles
         \footnotemark\addtocounter{footnote}{-1}\addtocounter{Hfootnote}{-1}
   \item[NPHIA] MPHIA + number of constraint torsions \footnotemark
   \item[NUMBND] Number of unique bond types
   \item[NUMANG] Number of unique angle types
   \item[NPTRA] Number of unique torsion types
   \item[NATYP] Number of SOLTY terms. Currently unused.
   \item[NPHB] Number of distinct 10-12 hydrogen bond pair types
      \footnote{Modern AMBER force fields do not use a 10-12 potential}
   \item[IFPERT] Set to 1 if topology contains residue perturbation information.
      \footnote{No AMBER codes support perturbed topologies anymore}
      \addtocounter{footnote}{-1}\addtocounter{Hfootnote}{-1}
   \item[NBPER] Number of perturbed bonds \footnotemark
      \addtocounter{footnote}{-1} \addtocounter{Hfootnote}{-1}
   \item[NGPER] Number of perturbed angles \footnotemark
      \addtocounter{footnote}{-1} \addtocounter{Hfootnote}{-1}
   \item[NDPER] Number of perturbed torsions \footnotemark
      \addtocounter{footnote}{-1} \addtocounter{Hfootnote}{-1}
   \item[MBPER] Number of bonds in which both atoms are being perturbed
      \footmarknote \addtocounter{footnote}{-1} \addtocounter{Hfootnote}{-1}
   \item[MGPER] Number of angles in which all 3 atoms are being perturbed
      \footmarknote \addtocounter{footnote}{-1} \addtocounter{Hfootnote}{-1}
   \item[MDPER] Number of torsions in which all 4 atoms are being perturbed
      \footnotemark
   \item[IFBOX] Flag indicating whether a periodic box is present. Values can be
      0 (no box), 1 (orthorhombic box) or 2 (truncated octahedron)
   \item[NMXRS] Number of atoms in the largest residue
   \item[IFCAP] Set to 1 if a solvent CAP is being used
   \item[NUMEXTRA] Number of extra points in the topology file
   \item[NCOPY] Number of PIMD slices or number of beads
\end{description}

\subsection*{ATOM\_NAME}

This section contains the atom name for every atom in the prmtop.

\noindent {\tt \%FORMAT(20a4)}
\noindent There are {\tt NATOM} atom names in this section.

\subsection*{CHARGE}

This section contains the charge for every atom in the prmtop. Charges are
multiplied by 18.2223 ($\sqrt{k_{ele}}$ where $k_{ele}$ is the electrostatic
constant in $kcal$ $\AA$ $mol^{-1}$ $q^{-2}$, where $q$ is the charge of an
electron).

\noindent {\tt \%FORMAT(5E16.8)}

\noindent There are {\tt NATOM} charges in this section.

\subsection*{ATOMIC\_NUMBER}

This section contains the atomic number of every atom in the prmtop. This
section was first introduced in AmberTools 12. \cite{AMBER12}

\noindent {\tt \%FORMAT(10I8)}

\noindent There are {\tt NATOM} atomic numbers in this section.

\subsection*{MASS}

This section contains the atomic mass of every atom in $g$ $mol^{-1}$.

\noindent {\tt \%FORMAT(5E16.8}

\noindent There are {\tt NATOM} atomic masses in this section.

\subsection*{ATOM\_TYPE\_INDEX}

This section contains the Lennard-Jones atom type index. The Lennard-Jones
potential contains parameters for every pair of atoms in the system. To minimize
the memory requirements of storing {\tt NATOM} $\times$ {\tt NATOM}
\footnote{Only half this number would be required, since $a_{i,j} \equiv
a_{j,i}$} Lennard-Jones A-coefficients and B-coefficients, all atoms with the
same $\sigma$ and $\varepsilon$ parameters are assigned to the same type
(regardless of whether they have the same {\tt AMBER\_ATOM\_TYPE}). This
significantly reduces the number of LJ coefficients which must be stored, but
introduced the requirement for bookkeeping sections of the topology file to keep
track of what the LJ type index was for each atom.

This section is used to compute a pointer into the {\tt NONBONDED\_PARM\_INDEX}
section, which itself is a pointer into the {\tt LENNARD\_JONES\_ACOEF} and {\tt
LENNARD\_JONES\_BCOEF} sections (see below).

\noindent {\tt \%FORMAT(10I8)}

\noindent There are {\tt NATOM} integers in this section.

\subsection*{NUMBER\_EXCLUDED\_ATOMS}

This section contains the number of atoms that need to be excluded from the
non-bonded calculation loop for atom $i$ because $i$ is involved in a bond,
angle, or torsion with those atoms. Each atom in the prmtop has a list of
excluded atoms that is a subset of the list in {\tt EXCLUDED\_ATOMS\_LIST} (see
below). The $i$th value in this section indicates how many elements of {\tt
EXCLUDED\_ATOMS\_LIST} belong to atom $i$.

For instance, if the first two elements of this array is 5 and 3, then elements
1 to 5 in {\tt EXCLUDED\_ATOMS\_LIST} are the exclusions for atom 1 and elements
6 to 8 in {\tt EXCLUDED\_ATOMS\_LIST} are the exclusions for atom 2. Each
exclusion is listed only once in the topology file, and is given to the atom
with the smaller index. That is, if atoms 1 and 2 are bonded, then atom 2 is in
the exclusion list for atom 1, but atom 1 is \emph{not} in the exclusion list
for atom 2. If an atom has no excluded atoms (either because it is a monoatomic
ion or all atoms it forms a bonded interaction with has a smaller index), then
it is given a value of 1 in this list which corresponds to an exclusion with (a
non-existent) atom 0 in {\tt EXCLUDED\_ATOMS\_LIST}.

The exclusion rules for extra points are more complicated. When determining
exclusions, it is considered an `extension' of the atom it is connected (bonded)
to. Therefore, extra points are excluded not only from the atom they are
connected to, but also from every atom that its parent atom is excluded from.

\emph{NOTE}: The non-bonded interaction code in \emph{sander} and \emph{pmemd}
currently (as of Amber 12) recalculates the exclusion lists for simulations of
systems with periodic boundary conditions, so this section is effectively
ignored. The GB code uses the exclusion list in the topology file.

\noindent {\tt \%FORMAT(10I8)}

\noindent There are {\tt NATOM} integers in this section.

\subsection*{NONBONDED\_PARM\_INDEX}

This section contains the pointers for each pair of LJ atom types into the {\tt
LENNARD\_JONES\_ACOEF} and {\tt LENNARD\_JONES\_BCOEF} arrays (see below). The
pointer for an atom pair in this array is calculated from the LJ atom type index
of the two atoms (see {\tt ATOM\_TYPE\_INDEX} above).

The index for two atoms $i$ and $j$ into the {\tt LENNARD\_JONES\_ACOEF} and
{\tt LENNARD\_JONES\_BCOEF} arrays is calculated as 
\begin{equation}
   index = {\tt NONBONDED\_PARM\_INDEX} \left [ {\tt NTYPES} \times \left ( {\tt
      ATOM\_TYPE\_INDEX}(i) - 1 \right ) + {\tt ATOM\_TYPE\_INDEX}(j) \right ]
   \label{eqB:ParmIndex}
\end{equation}

Note, each atom pair can interact with either the standard 12-6 LJ potential
\emph{or} via a 12-10 hydrogen bond potential. If $index$ in Eq.
\ref{eqB:ParmIndex} is negative, then it is an index into {\tt HBOND\_ACOEF} and
{\tt HBOND\_BCOEF} instead (see below).

\noindent {\tt \%FORMAT(10I8)}

\noindent There are ${\tt NTYPES} \times {\tt NTYPES}$ integers in this section.

\subsection*{RESIDUE\_LABEL}

This section contains the residue name for every residue in the prmtop. Residue
names are limited to 4 letters, and may not be whitespace-delimited if any
residues have 4-letter names.

\noindent {\tt \%FORMAT(20a4)}

\noindent There are {\tt NRES} 4-character strings in this section.

\subsection*{RESIDUE\_POINTER}

This section lists the first atom in each residue.

\noindent {\tt \%FORMAT(10i8)}

\noindent There are {\tt NRES} integers in this section.

\subsection*{BOND\_FORCE\_CONSTANT}

This section lists all of the bond force constants in units $kcal$ $mol^{-1}$
$\AA^{-2}$ for all unique bond types. Each bond in {\tt BONDS\_INC\_HYDROGEN}
and {\tt BONDS\_WITHOUT\_HYDROGEN} (see below) contains an index into this
array.

\noindent {\tt \%FORMAT(5E16.8)}

\noindent There are {\tt NUMBND} integers in this section.
array corresponding 
