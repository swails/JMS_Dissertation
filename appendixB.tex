\chapter{AMBER Parameter-Topology File Format}%
\label{appendixB}

This appendix details the Parameter-Topology file format used extensively by the
AMBER software suite for biomolecular simulation and analysis, referred to as
the \emph{prmtop} file for short. The format specification of the AMBER topology
file was written initially over a decade ago and posted on
http://ambermd.org/formats.html. I have recently expanded that document to
account for the drastic change to the file format that occurred with the 2004
release of Amber 7. The pre-Amber 7 format (\emph{old format}) is described more
briefly afterwards, although each section provided in the original format
contains exactly the same information as the newer version.

This appendix also details the format changes and additions introduced by
\emph{chamber}---the program that translates a CHARMM parameter file (PSF) into
a topology file that can be used with the \emph{sander} and \emph{pmemd}
programs in AMBER.

This appendix draws from the information on http://ambermd.org/formats.html that
was added by both me and others, as well as the experience I gleaned while
writing the ParmEd program and working with the various codes in AMBER.

As a warning, the prmtop file is a result of bookkeeping that becomes
increasingly complex as the system size increases. Therefore, hand-editing the
topology file for all but the smallest systems is discouraged---a program or
script should be written to automate the procedure.

\section{Layout}

The first line of the Amber topology file is the version string. An example is
shown below in which {\tt XX} is replaced by the actual date and time.
\begin{verbatim}
%VERSION  VERSION_STAMP = V0001.000  DATE = XX/XX/XX  XX:XX:XX
\end{verbatim}

The topology format is divided into several sections in a way that is designed
to be parsed easily using simple Fortran code. A consequence of this is that it
is difficult for parsers written in other languages (\eg C, C++, Python, etc.)
to strictly adhere to the standard. These parsers should try, however, to
support as much of the standard as possible.

\begin{verbatim}
%FLAG SECTION
%COMMENT an arbitrary number of optional comments may be put here
%FORMAT(<FORTRAN FORMAT>)
... data formatted according to <FORTRAN FORMAT>
\end{verbatim}

All names (\eg atom names, atom type names, and residue names) are limited to
4 characters and are printed in fields of width \emph{exactly} 4 characters
wide, left-justified. This means that names might not be space-delimited if any
of the names have 4 characters.

\paragraph{Requirements for prmtop parsers}

Parsers, regardless of the language they are written in, should conform to a
list of attributes to maximize the likelihood that they are parsed correctly.

\begin{itemize}
   \item Parsers should expect that some 4-character fields (\eg atom or residue
         names) may have some names that have 4 characters and therefore might
         not be whitespace-delimited.
   \item Parsers should not expect {\tt SECTION}s in the prmtop to be in any
         particular order.
   \item Parsers should not expect or require {\tt \%COMMENT} lines to exist,
         but should properly parse the file if any number of {\tt \%COMMENT}
         lines appear as indicated above
   \item The topology file may be assumed to have been generated `correctly' by
         \emph{tleap} or some other credible source. No graceful error checking
         is required.
\end{itemize}

\paragraph{Requirements for modifying {\tt SECTION}s}

To minimize the impact of prmtop changes to existing, third-party parsers, the
following conventions should be followed.

\begin{itemize}
   \item Any new {\tt SECTION} should be added to the end of the topology file to
         avoid conflicts with order-dependent parsers.
   \item The {\tt <FORTRAN FORMAT>} should be as simple as possible (and avoid
         adding new formats) to maintain simplicity for non-Fortran parsers.
   \item Avoid modifying if possible. Consider if this new section or change is
         truly necessary \emph{and} belongs in the prmtop.
\end{itemize}

\section{List of {\tt SECTION}s}

\subsection*{TITLE}

This section contains the title of the topology file on one line (up to 80
characters). While the title serves a primarily cosmetic purpose, this section
must be present.

\noindent {\tt \%FORMAT(20a4)}

\subsection*{POINTERS}

This section contains the information about how many parameters are present in
all of the sections. There are 31 or 32 integer pointers (NCOPY might not be
present). The format and names of all of the pointers are listed below, followed
by a description of each pointer.

\vspace{8pt}
\hline
\vspace{4pt}

\begin{verbatim}
%FLAG POINTERS
%FORMAT(10I8)
NATOM  NTYPES NBONH  MBONA  NTHETH MTHETA NPHIH  MPHIA  NHPARM NPARM  
NNB    NRES   NBONA  NTHETA NPHIA  NUMBND NUMANG NPTRA  NATYP  NPHB   
IFPERT NBPER  NGPER  NDPER  MBPER  MGPER  MDPER  IFBOX  NMXRS  IFCAP  
NUMEXTRA NCOPY 
\end{verbatim}

\vspace{4pt}
\hline
\vspace{8pt}

\begin{description}
   \item[NATOM] Number of atoms
   \item[NTYPES] Number of distinct Lennard-Jones atom types
   \item[NBONH] Number of bonds containing Hydrogen
   \item[MBONA] Number of bonds not containing Hydrogen
   \item[NTHETH] Number of angles containing Hydrogen
   \item[MTHETA] Number of angles not containing Hydrogen
   \item[NPHIH] Number of torsions containing Hydrogen
   \item[MPHIA] Number of torsions not containing Hydrogen
   \item[NHPARM] Not currently used for anything
   \item[NPARM] Used to determine if this is a LES-compatible prmtop
   \item[NNB] Number of excluded atoms (length of total exclusion list)
   \item[NRES] Number of residues
   \item[NBONA] MBONA + number of constraint bonds \footnote{AMBER codes no
         longer support constraints in the topology
         file.}\addtocounter{footnote}{-1}\addtocounter{Hfootnote}{-1}
   \item[NTHETA] MTHETA + number of constraint angles
         \footnotemark\addtocounter{footnote}{-1}\addtocounter{Hfootnote}{-1}
   \item[NPHIA] MPHIA + number of constraint torsions \footnotemark
   \item[NUMBND] Number of unique bond types
   \item[NUMANG] Number of unique angle types
   \item[NPTRA] Number of unique torsion types
   \item[NATYP] Number of SOLTY terms. Currently unused.
   \item[NPHB] Number of distinct 10-12 hydrogen bond pair types
      \footnote{Modern AMBER force fields do not use a 10-12 potential}
   \item[IFPERT] Set to 1 if topology contains residue perturbation information.
      \footnote{No AMBER codes support perturbed topologies anymore}
      \addtocounter{footnote}{-1}\addtocounter{Hfootnote}{-1}
   \item[NBPER] Number of perturbed bonds \footnotemark
      \addtocounter{footnote}{-1} \addtocounter{Hfootnote}{-1}
   \item[NGPER] Number of perturbed angles \footnotemark
      \addtocounter{footnote}{-1} \addtocounter{Hfootnote}{-1}
   \item[NDPER] Number of perturbed torsions \footnotemark
      \addtocounter{footnote}{-1} \addtocounter{Hfootnote}{-1}
   \item[MBPER] Number of bonds in which both atoms are being perturbed
      \footmarknote \addtocounter{footnote}{-1} \addtocounter{Hfootnote}{-1}
   \item[MGPER] Number of angles in which all 3 atoms are being perturbed
      \footmarknote \addtocounter{footnote}{-1} \addtocounter{Hfootnote}{-1}
   \item[MDPER] Number of torsions in which all 4 atoms are being perturbed
      \footnotemark
   \item[IFBOX] Flag indicating whether a periodic box is present. Values can be
      0 (no box), 1 (orthorhombic box) or 2 (truncated octahedron)
   \item[NMXRS] Number of atoms in the largest residue
   \item[IFCAP] Set to 1 if a solvent CAP is being used
   \item[NUMEXTRA] Number of extra points in the topology file
   \item[NCOPY] Number of PIMD slices or number of beads
\end{description}

\subsection*{ATOM\_NAME}

This section contains the atom name for every atom in the prmtop.

\noindent {\tt \%FORMAT(20a4)}
\noindent There are {\tt NATOM} 4-character strings in this section.

\subsection*{CHARGE}

This section contains the charge for every atom in the prmtop. Charges are
multiplied by 18.2223 ($\sqrt{k_{ele}}$ where $k_{ele}$ is the electrostatic
constant in $kcal\, \AA\, mol^{-1}\, q^{-2}$, where $q$ is the charge of an
electron).

\noindent {\tt \%FORMAT(5E16.8)}

\noindent There are {\tt NATOM} floating point numbers in this section.

\subsection*{ATOMIC\_NUMBER}

This section contains the atomic number of every atom in the prmtop. This
section was first introduced in AmberTools 12. \cite{AMBER12}

\noindent {\tt \%FORMAT(10I8)}

\noindent There are {\tt NATOM} integers in this section.

\subsection*{MASS}

This section contains the atomic mass of every atom in $g\, mol^{-1}$.

\noindent {\tt \%FORMAT(5E16.8)}

\noindent There are {\tt NATOM} floating point numbers in this section.

\subsection*{ATOM\_TYPE\_INDEX}

This section contains the Lennard-Jones atom type index. The Lennard-Jones
potential contains parameters for every pair of atoms in the system. To minimize
the memory requirements of storing {\tt NATOM} $\times$ {\tt NATOM}
\footnote{Only half this number would be required, since $a_{i,j} \equiv
a_{j,i}$} Lennard-Jones A-coefficients and B-coefficients, all atoms with the
same $\sigma$ and $\varepsilon$ parameters are assigned to the same type
(regardless of whether they have the same {\tt AMBER\_ATOM\_TYPE}). This
significantly reduces the number of LJ coefficients which must be stored, but
introduced the requirement for bookkeeping sections of the topology file to keep
track of what the LJ type index was for each atom.

This section is used to compute a pointer into the {\tt NONBONDED\_PARM\_INDEX}
section, which itself is a pointer into the {\tt LENNARD\_JONES\_ACOEF} and {\tt
LENNARD\_JONES\_BCOEF} sections (see below).

\noindent {\tt \%FORMAT(10I8)}

\noindent There are {\tt NATOM} integers in this section.

\subsection*{NUMBER\_EXCLUDED\_ATOMS}

This section contains the number of atoms that need to be excluded from the
non-bonded calculation loop for atom $i$ because $i$ is involved in a bond,
angle, or torsion with those atoms. Each atom in the prmtop has a list of
excluded atoms that is a subset of the list in {\tt EXCLUDED\_ATOMS\_LIST} (see
below). The $i$th value in this section indicates how many elements of {\tt
EXCLUDED\_ATOMS\_LIST} belong to atom $i$.

For instance, if the first two elements of this array is 5 and 3, then elements
1 to 5 in {\tt EXCLUDED\_ATOMS\_LIST} are the exclusions for atom 1 and elements
6 to 8 in {\tt EXCLUDED\_ATOMS\_LIST} are the exclusions for atom 2. Each
exclusion is listed only once in the topology file, and is given to the atom
with the smaller index. That is, if atoms 1 and 2 are bonded, then atom 2 is in
the exclusion list for atom 1, but atom 1 is \emph{not} in the exclusion list
for atom 2. If an atom has no excluded atoms (either because it is a monoatomic
ion or all atoms it forms a bonded interaction with have a smaller index), then
it is given a value of 1 in this list which corresponds to an exclusion with (a
non-existent) atom 0 in {\tt EXCLUDED\_ATOMS\_LIST}.

The exclusion rules for extra points are more complicated. When determining
exclusions, it is considered an `extension' of the atom it is connected (bonded)
to. Therefore, extra points are excluded not only from the atom they are
connected to, but also from every atom that its parent atom is excluded from.

\emph{NOTE}: The non-bonded interaction code in \emph{sander} and \emph{pmemd}
currently (as of Amber 12) recalculates the exclusion lists for simulations of
systems with periodic boundary conditions, so this section is effectively
ignored. The GB code uses the exclusion list in the topology file.

\noindent {\tt \%FORMAT(10I8)}

\noindent There are {\tt NATOM} integers in this section.

\subsection*{NONBONDED\_PARM\_INDEX}

\sloppy
This section contains the pointers for each pair of LJ atom types into the {\tt
LENNARD\_JONES\_ACOEF} and {\tt LENNARD\_JONES\_BCOEF} arrays (see below). The
pointer for an atom pair in this array is calculated from the LJ atom type index
of the two atoms (see {\tt ATOM\_TYPE\_INDEX} above).

The index for two atoms $i$ and $j$ into the {\tt LENNARD\_JONES\_ACOEF} and
{\tt LENNARD\_JONES\_BCOEF} arrays is calculated as 
\begin{equation}
   index = {\tt NONBONDED\_PARM\_INDEX} \left [ {\tt NTYPES} \times \left ( {\tt
      ATOM\_TYPE\_INDEX}(i) - 1 \right ) + {\tt ATOM\_TYPE\_INDEX}(j) \right ]
   \label{eqB:ParmIndex}
\end{equation}

Note, each atom pair can interact with either the standard 12-6 LJ potential
\emph{or} via a 12-10 hydrogen bond potential. If $index$ in Eq.
\ref{eqB:ParmIndex} is negative, then it is an index into {\tt HBOND\_ACOEF} and
{\tt HBOND\_BCOEF} instead (see below).

\noindent {\tt \%FORMAT(10I8)}

\noindent There are ${\tt NTYPES} \times {\tt NTYPES}$ integers in this section.

\subsection*{RESIDUE\_LABEL}

This section contains the residue name for every residue in the prmtop. Residue
names are limited to 4 letters, and might not be whitespace-delimited if any
residues have 4-letter names.

\noindent {\tt \%FORMAT(20a4)}

\noindent There are {\tt NRES} 4-character strings in this section.

\subsection*{RESIDUE\_POINTER}

This section lists the first atom in each residue.

\noindent {\tt \%FORMAT(10i8)}

\noindent There are {\tt NRES} integers in this section.

\subsection*{BOND\_FORCE\_CONSTANT}

Bond energies are calculated according to the equation
\begin{equation}
   E_{bond} = \frac 1 2 k \left ( \vec{r} - \vec{r}_{eq} \right ) ^ 2
   \label{eqB:Bond}
\end{equation}

This section lists all of the bond force constants ($k$ in Eq. \ref{eqB:Bond})
in units $kcal\, mol^{-1}\, \AA^{-2}$ for each unique bond type. Each bond in
{\tt BONDS\_INC\_HYDROGEN} and {\tt BONDS\_WITHOUT\_HYDROGEN} (see below)
contains an index into this array.

\noindent {\tt \%FORMAT(5E16.8)}

\noindent There are {\tt NUMBND} floating point numbers in this section.

\subsection*{BOND\_EQUIL\_VALUE}

This section lists all of the bond equilibrium distances ($\vec{r}_{eq}$ in Eq.
\ref{eqB:Bond}) in units of {\AA} for each unique bond type. This list is
indexed the same way as {\tt BOND\_FORCE\_CONSTANT}.

\noindent {\tt \%FORMAT(5E16.8)}

\noindent There are {\tt NUMBND} floating point numbers in this section.

\subsection*{ANGLE\_FORCE\_CONSTANT}

Angle energies are calculated according to the equation
\begin{equation}
   E_{angle} = \frac 1 2 k_{\theta} \left ( \theta - \theta _ {eq} \right ) ^ 2
   \label{eqB:Angle}
\end{equation}

This section lists all of the angle force constants ($k_{\theta}$ in Eq.
\ref{eqB:Angle}) in units of $kcal\, mol^{-1}\, rad^2$ for each unique angle
type. Each angle in {\tt ANGLES\_INC\_HYDROGEN} and {\tt
ANGLES\_WITHOUT\_HYDROGEN} contains an index into this (and the next) array.

\noindent {\tt \%FORMAT(5E16.8)}

\noindent There are {\tt NUMANG} floating point numbers in this section.

\subsection*{ANGLE\_EQUIL\_VALUE}

This section contains all of the angle equilibrium angles ($\theta_{eq}$ in Eq.
\ref{eqB:Angle}) in radians.  \emph{NOTE:} the AMBER parameter files list
equilibrium angles in degrees and are converted to radians in \emph{tleap}. This
list is indexed the same way as {\tt ANGLE\_FORCE\_CONSTANT}.

\noindent {\tt \%FORMAT(5E16.8)}

\noindent There are {\tt NUMBND} floating point numbers in this section.

\subsection*{DIHEDRAL\_FORCE\_CONSTANT}

Torsion energies are calculated for each term according to the equation
\begin{equation}
   E_{torsion} = k_{tor} \cos \left ( n \phi + \psi \right )
   \label{eqB:Dihedral}
\end{equation}

\sloppy
This section lists the torsion force constants ($k_{tor}$ in Eq.
\ref{eqB:Dihedral}) in units of $kcal\, mol^{-1}$ for each unique torsion type.
Each torsion in {\tt DIHEDRALS\_INC\_HYDROGEN} and {\tt
DIHEDRALS\_WITHOUT\_HYDROGEN} has an index into this array.

Amber parameter files contain a dividing factor and barrier height for each
dihedral. The barrier height in the parameter files are divided by the provided
factor inside \emph{tleap} and then discarded. As a result, the torsion barriers
in this section might not match those in the original parameter files.

\noindent {\tt \%FORMAT(5E16.8)}

\noindent There are {\tt NPTRA} floating point numbers in this section.

\subsection*{DIHEDRAL\_PERIODICITY}

This section lists the periodicity ($n$ in Eq. \ref{eqB:Dihedral}) for each
unique torsion type. It is indexed the same way as {\tt
DIHEDRAL\_FORCE\_CONSTANT}. \emph{NOTE:} only integers are read by
\emph{tleap}, although the AMBER codes support non-integer periodicities.

\noindent {\tt \%FORMAT(5E16.8)}

\noindent There are {\tt NPTRA} floating point numbers in this section.

\subsection*{DIHEDRAL\_PHASE}

This section lists the phase shift ($\psi$ in Eq. \ref{eqB:Dihedral}) for each
unique torsion type in radians. It is indexed the same way as {\tt
DIHEDRAL\_FORCE\_CONSTANT}.

\noindent {\tt \%FORMAT(5E16.8)}

\noindent There are {\tt NPTRA} floating point numbers in this section.

\subsection*{SCEE\_SCALE\_FACTOR}

This section was introduced in Amber 11. In previous versions, this variable was
part of the input file and set a single scaling factor for every torsion.

This section lists the factor by which 1-4 electrostatic interactions are
divided (\ie the two atoms on either end of a torsion). For torsion types in
which 1-4 non-bonded interactions are not calculated (\eg improper torsions,
multi-term torsions, and those involved in ring systems of 6 or fewer atoms), a
value of 0 is assigned by \emph{tleap}. This section is indexed the same way as
{\tt DIHEDRAL\_FORCE\_CONSTANT}.

\noindent {\tt \%FORMAT(5E16.8)}

\noindent There are {\tt NPTRA} floating point numbers in this section.

\subsection*{SCNB\_SCALE\_FACTOR}

This section was introduced in Amber 11. In previous versions, this variable was
part of the input file and set a single scaling factor for every torsion.

This section lists the factor by which 1-4 van der Waals interactions are
divided (\ie the two atoms on either end of a torsion). This section is
analogous to {\tt SCEE\_SCALE\_FACTOR} described above.

\noindent {\tt \%FORMAT(5E16.8)}

\noindent There are {\tt NPTRA} floating point numbers in this section.

\subsection*{SOLTY}

This section is currently unused, and while `future use' is planned, this
assertion has lain dormant for some time.

\noindent {\tt \%FORMAT(5E16.8)}

\noindent There are {\tt NATYP} floating point numbers in this section.

\subsection*{LENNARD\_JONES\_ACOEF}

LJ non-bonded interactions are calculated according to the equation
\begin{equation}
   E_{LJ} = \frac {a_{i,j}} {r ^ {12}} - \frac {b_{i,j}} {r ^ 6}
   \label{eqB:LennardJones}
\end{equation}

This section contains the LJ A-coefficients ($a_{i,j}$ in Eq.
\ref{eqB:LennardJones}) for all pairs of distinct LJ types (see sections {\tt
ATOM\_TYPE\_INDEX} and {\tt NONBONDED\_PARM\_INDEX} above).

\noindent {\tt \%FORMAT(5E16.8)}

\noindent There are $\left [ {\tt NTYPES} \times ( {\tt NTYPES} + 1 ) \right ] /
2$ floating point numbers in this section.

\subsection*{LENNARD\_JONES\_BCOEF}

This section contains the LJ B-coefficients ($b_{i,j}$ in Eq.
\ref{eqB:LennardJones}) for all pairs of distinct LJ types (see sections {\tt
ATOM\_TYPE\_INDEX} and {\tt NONBONDED\_PARM\_INDEX} above).

\noindent {\tt \%FORMAT(5E16.8)}

\noindent There are $\left [ {\tt NTYPES} \times ( {\tt NTYPES} + 1 ) \right ] /
2$ floating point numbers in this section.

\subsection*{BONDS\_INC\_HYDROGEN}

This section contains a list of every bond in the system in which at least one
atom is Hydrogen. Each bond is identified by 3 integers---the two atoms involved
in the bond and the index into the {\tt BOND\_FORCE\_CONSTANT} and {\tt
BOND\_EQUIL\_VALUE}. For run-time efficiency, the atom indexes are actually
indexes into a coordinate array, so the actual atom index $A$ is calculated from
the coordinate array index $N$ by $A = N / 3 + 1$. ($N$ is the value in the
topology file)

\noindent {\tt \%FORMAT(10I8)}

\noindent There are $3 \times {\tt NBONH}$ integers in this section.

\subsection*{BONDS\_WITHOUT\_HYDROGEN}

This section contains a list of every bond in the system in which neither atom
is Hydrogen. It has the same structure as {\tt BONDS\_INC\_HYDROGEN} described
above.

\noindent {\tt \%FORMAT(10I8)}

\noindent There are $3 \times {\tt NBONA}$ integers in this section.

\subsection*{ANGLES\_INC\_HYDROGEN}

This section contains a list of every angle in the system in which at least one
atom is Hydrogen. Each angle is identified by 4 integers---the three atoms
involved in the angle and the index into the {\tt ANGLE\_FORCE\_CONSTANT} and
{\tt ANGLE\_EQUIL\_VALUE}. For run-time efficiency, the atom indexes are
actually indexes into a coordinate array, so the actual atom index $A$ is
calculated from the coordinate array index $N$ by $A = N / 3 + 1$. ($N$ is the
value in the topology file)

\noindent {\tt \%FORMAT(10I8)}

\noindent There are $4 \times {\tt NTHETH}$ integers in this section.

\subsection*{ANGLES\_WITHOUT\_HYDROGEN}

This section contains a list of every angle in the system in which no atom is
Hydrogen. It has the same structure as {\tt ANGLES\_INC\_HYDROGEN} described
above.

\noindent {\tt \%FORMAT(10I8)}

\noindent There are $4 \times {\tt NTHETA}$ integers in this section.

\subsection*{DIHEDRALS\_INC\_HYDROGEN}

This section contains a list of every torsion in the system in which at least
one atom is Hydrogen. Each torsion is identified by 5 integers---the four atoms
involved in the torsion and the index into the {\tt DIHEDRAL\_FORCE\_CONSTANT},
{\tt DIHEDRAL\_PERIODICITY}, {\tt DIHEDRAL\_PHASE}, {\tt SCEE\_SCALE\_FACTOR}
and {\tt SCNB\_SCALE\_FACTOR} arrays. For run-time efficiency, the atom indexes
are actually indexes into a coordinate array, so the actual atom index $A$ is
calculated from the coordinate array index $N$ by $A = N / 3 + 1$. ($N$ is the
value in the topology file)

If the third atom is negative, then the 1-4 non-bonded interactions for this
torsion is not calculated. This is required to avoid double-counting these
non-bonded interactions in some ring systems and in multi-term torsions.

If the fourth atom is negative, then the torsion is improper.

\emph{NOTE:} The first atom has an index of zero. Since 0 cannot be negative and
the 3rd and 4th atom indexes are tested for their sign to determine if 1-4 terms
are calculated, the first atom in the topology file must be listed as either the
first or second atom in whatever torsions it is defined in. The atom ordering in
a torsion can be reversed to accommodate this requirement if necessary.

\noindent {\tt \%FORMAT(10I8)}

\noindent There are $5 \times {\tt NPHIH}$ integers in this section.

\subsection*{DIHEDRALS\_WITHOUT\_HYDROGEN}

This section contains a list of every torsion in the system in which no atom is
Hydrogen. It has the same structure as {\tt DIHEDRALS\_INC\_HYDROGEN} described
above.

\noindent {\tt \%FORMAT(10I8)}

\noindent There are $5 \times {\tt NPHIA}$ integers in this section.

\subsection*{EXCLUDED\_ATOMS\_LIST}

This section contains a list for each atom of excluded partners in the
non-bonded calculation routines. The subset of this list that belongs to each
atom is determined from the pointers in {\tt NUMBER\_EXCLUDED\_ATOMS}---see that
section for more information.

\emph{NOTE:} The periodic boundary code in \emph{sander} and \emph{pmemd}
currently recalculates this section of the topology file. The GB code, however,
uses the exclusion list defined in the topology file.

\noindent {\tt \%FORMAT(10I8)}

\noindent There are {\tt NNB} integers in this section.

\subsection*{HBOND\_ACOEF}

This section is analogous to the {\tt LENNARD\_JONES\_ACOEF} array described
above, but refers to the A-coefficient in a 12-10 potential instead of the
familiar 12-6 potential. This term has been dropped from most modern force
fields.

\noindent {\tt \%FORMAT(5E16.8)}

\noindent There are {\tt NPHB} floating point numbers in this section.

\subsection*{HBOND\_BCOEF}

This section is analogous to the {\tt LENNARD\_JONES\_BCOEF} array described
above, but refers to the B-coefficient in a 12-10 potential instead of the
familiar 12-6 potential. This term has been dropped from most modern force
fields.

\noindent {\tt \%FORMAT(5E16.8)}

\noindent There are {\tt NPHB} floating point numbers in this section.

\subsection*{HBCUT}

This section used to be used for a cutoff parameter in the 12-10 potential, but
is no longer used for anything.

\noindent {\tt \%FORMAT(5E16.8)}

\noindent There are {\tt NPHB} floating point numbers in this section.

\subsection*{AMBER\_ATOM\_TYPE}

This section contains the atom type name for every atom in the prmtop.

\noindent {\tt \%FORMAT(20a4)}

\noindent There are {\tt NATOM} 4-character strings in this section.

\subsection*{TREE\_CHAIN\_CLASSIFICATION}

This section contains information about the tree structure (borrowing concepts
from \emph{graph theory}) of each atom. Each atom can have one of the following
character indicators:
\begin{description}
   \item[M] This atom is part of the ``main chain''
   \item[S] This atom is part of the ``sidechain''
   \item[E] This atom is a chain-terminating atom (\ie an ``end'' atom)
   \item[3] The structure branches into 3 chains at this point
   \item[BLA] If none of the above are true
\end{description}

\noindent {\tt \%FORMAT(20a4)}

\noindent There are {\tt NATOM} 4-character strings in this section.

\subsection*{JOIN\_ARRAY}

This section is no longer used and is currently just filled with zeros.

\noindent {\tt \%FORMAT(10I8)}

\noindent There are {\tt NATOM} integers in this section.

\subsection*{IROTAT}

This section is not used and is currently just filled with zeros.

\noindent {\tt \%FORMAT(10I8)}

\noindent There are {\tt NATOM} integers in this section.

\subsection*{SOLVENT\_POINTERS}

This section is only present if {\tt IFBOX} is greater than 0 (\ie if the system
was set up for use with periodic boundary conditions). There are 3 integers
present in this section---the final residue that is part of the solute ({\tt
IPTRES}), the total number of `molecules' ({\tt NSPM}), and the first solvent
`molecule' ({\tt NSPSOL}).

A `molecule' is defined as a closed graph---that is, there is a pathway from
every atom in a molecule to every other atom in the molecule by traversing
bonds, and there are no pathways to `other' molecules.

\begin{verbatim}
%FLAG SOLVENT_POINTERS
%FORMAT(3I8)
IPTRES   NSPM    NSPSOL
\end{verbatim}

\subsection*{ATOMS\_PER\_MOLECULE}

This section is only present if {\tt IFBOX} is greater than 0 (\ie if the system
was set up for use with periodic boundary conditions). This section lists how
many atoms are present in each `molecule' as defined in the {\tt
SOLVENT\_POINTERS} section above.

\noindent {\tt \%FORMAT(10I8)}

\noindent There are {\tt NSPM} integers in this section (see the {\tt
SOLVENT\_POINTERS} section above).

\subsection*{BOX\_DIMENSIONS}

This section is only present if {\tt IFBOX} is greater than 0 (\ie if the system
was set up for use with periodic boundary conditions). This section lists the
box angle ({\tt OLDBETA}) and dimensions (${\tt BOX(1)} \times {\tt BOX(2)}
\times {\tt BOX(3)}$). The values in this section are deprecated now since newer
and more accurate information about the box size and shape is stored in the
coordinate file. Since constant pressure simulations can change the box
dimensions, the values in the coordinate file should be trusted over those in
the topology file.

\begin{verbatim}
%FLAG BOX_DIMENSIONS
%FORMAT(5E16.8)
OLDBETA         BOX(1)          BOX(2)          BOX(3)
\end{verbatim}

\subsection*{CAP\_INFO}

This section is present only if {\tt IFCAP} is not 0. If present, it contains a
single integer which is the last atom before the water cap begins ({\tt NATCAP})

\noindent {\tt \%FORMAT(10I8)}

\subsection*{CAP\_INFO2}

This section is present only if {\tt IFCAP} is not 0. If present, it contains
four numbers---the distance from the center of the cap to outside the cap ({\tt
CUTCAP}), and the Cartesian coordinates of the cap center.

\begin{verbatim}
%FLAG CAP_INFO2
%FORMAT(5E16.8)
CUTCAP          XCAP            YCAP            ZCAP
\end{verbatim}

\subsection*{RADIUS\_SET}

This section contains a one-line string (up to 80 characters) describing the
intrinsic implicit solvent radii set that are defined in the topology file. The
available radii sets with their 1-line descriptions are:
\begin{description}
   \item[bondi] {\tt Bondi radii (bondi)}
   \item[amber6] {\tt amber6 modified Bondi radii (amber6)}
   \item[mbondi] {\tt modified Bondi radii (mbondi)}
   \item[mbondi2] {\tt H(N)-modified Bondi radii (mbondi2)}
   \item[mbondi3] {\tt ArgH and AspGlu0 modified Bondi2 radii (mbondi3)}
\end{description}

\noindent {\tt \%FORMAT(1a80)}

\noindent There is a single line description in this section.

\subsection*{RADII}

This section contains the intrinsic radii of every atom used for implicit
solvent calculations (typically Generalized Born).

\noindent {\tt \%FORMAT(5E16.8)}

\noindent There are {\tt NATOM} floating point numbers in this section.

\subsection*{IPOL}

This section was introduced in Amber 12. In previous versions of Amber, this was
a variable in the input file.

This section contains a single integer that is 0 for fixed-charge force fields
and 1 for force fields that contain polarization.

\subsection*{POLARIZABILITY}

This section is only present if {\tt IPOL} is not 0. It contains the atomic
polarizabilities for every atom in the prmtop.

\noindent {\tt \%FORMAT(5E16.8)}

\noindent There are {\tt NATOM} floating point numbers in this section.

\noindent {\tt \%FORMAT(1I8)}

\section{Deprecated Sections}

All of the sections of the topology file listed here are only present if {\tt
IFPERT} is 1. However, no modern programs support such prmtops so these sections
are rarely (if ever) used. They are included in Table \ref{tblB:PertSections}
for completeness, only.

More info can be found online at http://ambermd.org/formats.html

\begin{table}
   \caption{List of all of the perturbed topology file sections.}
   \begin{tabular}{cccc}
      \hline
      FLAG name & {\tt \%FORMAT} & # of values & Description \\
      \hline
      PERT\_BOND\_ATOMS & {\tt 10I8} &  $2 \times {\tt NBPER}$ & perturbed bond
         list \\
      PERT\_BOND\_PARAMS & {\tt 10I8} & $2 \times {\tt NBPER}$ & perturbed bond
         pointers \\
      PERT\_ANGLE\_ATOMS & {\tt 10I8} & $3 \times {\tt NGPER}$ & perturbed angle
         list \\
      PERT\_ANGLE\_PARAMS & {\tt 10I8} & $2 \times {\tt NGPER}$ & perturbed
         angle pointers \\
      PERT\_DIHEDRAL\_ATOMS & {\tt 10I8} & $4 \times {\tt NDPER}$ & perturbed
         torsion list \\
      PERT\_DIHEDRAL\_PARAMS & {\tt 10I8} & $2 \times {\tt NDPER}$ & perturbed
         torsion pointers \\
      PERT\_RESIDUE\_NAME & {\tt 20a4} & {\tt NRES} & end state residue names \\
      PERT\_ATOM\_NAME & {\tt 20a4} & {\tt NATOM} & end state atom names \\
      PERT\_ATOM\_SYMBOL & {\tt 20a4} & {\tt NATOM} & end state atom types \\
      ALMPER & {\tt 5E16.8} & {\tt NATOM} & Unused \\
      IAPER & {\tt 10I8} & {\tt NATOM} & Is Atom PERturbed? \\
      PERT\_ATOM\_TYPE\_INDEX & {\tt 10I8} & {\tt NATOM} &
         Perturbed LJ Type \\
      PERT\_CHARGE & {\tt 5E16.8} & {\tt NATOM} & Perturbed charge \\
      \hline
   \end{tabular}
   \label{tblB:PertSections}
\end{table}

\section{CHAMBER Topologies}

Here we will describe the general format of topology files generated by the
\emph{chamber} program. The \emph{chamber} program was developed to translate
CHARMM topology (PSF) files into Amber topology files for use with the AMBER
program suite.

Due to differences in the CHARMM force field (\eg the extra CMAP and
Urey-Bradley terms and the different way that improper dihedrals are treated),
chamber topologies contain more sections than Amber topologies. Furthermore, to
ensure rigorous reproduction of CHARMM energies inside the AMBER program suites,
some of the sections that are common between AMBER and CHARMM topology files
have a different format for their data to support a different level of
input data precision.

Due to the differences in the \emph{chamber} topology files, a mechanism to
differentiate between \emph{chamber} topologies and AMBER topologies was
introduced. If the topology file has a {\tt \%FLAG TITLE} then it is an AMBER
topology. If it has a {\tt \%FLAG CTITLE} instead, then it is a \emph{chamber}
topology.

The following sections of the \emph{chamber} topology are exacly the same as
those from the AMBER topology files:
\begin{itemize}
   \item POINTERS
   \item ATOM\_NAME
   \item MASS
   \item ATOM\_TYPE\_INDEX
   \item NUMBER\_EXCLUDED\_ATOMS
   \item EXCLUDED\_ATOMS\_LIST
   \item NONBONDED\_PARM\_INDEX
   \item RESIDUE\_LABEL
   \item BOND\_FORCE\_CONSTANT
   \item BOND\_EQUIL\_VALUE
   \item ANGLE\_FORCE\_CONSTANT
   \item DIHEDRAL\_FORCE\_CONSTANT
   \item DIHEDRAL\_PERIODICITY
   \item DIHEDRAL\_PHASE
   \item SCEE\_SCALE\_FACTOR
   \item SCNB\_SCALE\_FACTOR
   \item SOLTY
   \item BONDS\_INC\_HYDROGEN
   \item BONDS\_WITHOUT\_HYDROGEN
   \item ANGLES\_INC\_HYDROGEN
   \item ANGLES\_WITHOUT\_HYDROGEN
   \item DIHEDRALS\_INC\_HYDROGEN
   \item DIHEDRALS\_WITHOUT\_HYDROGEN
   \item HBOND\_ACOEF
   \item HBOND\_BCOEF
   \item HBCUT
   \item AMBER\_ATOM\_TYPE
   \item TREE\_CHAIN\_CLASSIFICATION \footnote{Not really supported. Every entry
         is BLA}
   \item JOIN\_ARRAY
   \item IROTAT
   \item RADIUS\_SET
   \item RADII
   \item SCREEN
   \item SOLVENT\_POINTERS
   \item ATOMS\_PER\_MOLECULE
\end{itemize}

In Table \ref{tblB:ChamberDiffs} is a list of sections that have the same name
and the same data, but with a different Fortran format identifier.

\begin{table*}
   \caption{List of flags that are common between Amber and chamber topology
            files, but have different {\tt FORMAT} identifiers.}
   \begin{tabular}{ccc}
      \hline
      FLAG name & AMBER Format & chamber Format \\
      \hline
      CHARGE & {\tt 5E16.8} & {\tt 3E24.16} \\
      ANGLE\_EQUIL\_VALUE & {\tt 5E16.8} & {\tt 3E25.17} \\
      LENNARD\_JONES\_ACOEF & {\tt 5E16.8} & {\tt 3E24.16} \\
      LENNARD\_JONES\_BCOEF & {\tt 5E16.8} & {\tt 3E24.16} \\
      \hline
   \end{tabular}
   \label{tblB:ChamberDiffs}
\end{table*}

\subsection*{FORCE\_FIELD\_TYPE}

This section is a description of the CHARMM force field that is parametrized in
the topology file. It is a single line (it can be read as a single string of
length 80 characters). It does not affect any numerical results.

\noindent {\tt \%FORMAT(i2,a78)}

\subsection*{CHARMM\_UREY\_BRADLEY\_COUNT}

This section contains the number of Urey-Bradley parameters printed in the
topology file. It contains two integers, the total number of Urey-Bradley terms
({\tt NUB}) and the number of unique Urey-Bradley types ({\tt NUBTYPES}).

\begin{verbatim}
%FLAG CHARMM_UREY_BRADLEY_COUNT
%FORMAT(2i8)
NUB     NUBTYPES
\end{verbatim}

\subsection*{CHARMM\_UREY\_BRADLEY}

This section contains all of the Urey-Bradley terms. It is formatted exactly
like {\tt BONDS\_INC\_HYDROGEN} and {\tt BONDS\_WITHOUT\_HYDROGEN}.

\noindent {\tt \%FORMAT(10i8)}

\noindent There are $3 \times {\tt NUB}$ integers in this section.

\subsection*{CHARMM\_UREY\_BRADLEY\_FORCE\_CONSTANT}

This section contains all of the force constants for each unique Urey-Bradley
term in $kcal\, mol^{-1}\, \AA^2$. It is formatted exactly the same as {\tt
BOND\_FORCE\_CONSTANT}.

\noindent {\tt \%FORMAT(5E16.8)}

\noindent There are {\tt NUBTYPES} floating point numbers in this section.

\subsection*{CHARMM\_UREY\_BRADLEY\_EQUIL\_VALUE}

This section contains all of the equilibrium distances for each unique
Urey-Bradley term in \AA. It is formatted exactly the same as {\tt
BOND\_EQUIL\_VALUE}.

\noindent {\tt \%FORMAT(5E16.8)}

\noindent There are {\tt NUBTYPES} floating point numbers in this section.

\subsection*{CHARMM\_NUM\_IMPROPERS}

This section contains the number of improper torsions in the topology file.
It contains one integer, the total number of improper torsions.

\begin{verbatim}
%FLAG CHARMM_NUM_IMPROPERS
%FORMAT(i8)
NIMPHI
\end{verbatim}

\subsection*{CHARMM\_IMPROPERS}

This section contains all of the improper torsion terms. It is formatted exactly
like {\tt DIHEDRALS\_INC\_HYDROGEN} and {\tt DIHEDRALS\_WITHOUT\_HYDROGEN}.

\noindent {\tt \%FORMAT(10i8)}

\noindent There are $5 \times {\tt NIMPHI}$ integers in this section.

\subsection*{CHARMM\_NUM\_IMPROPER\_TYPES}

This section contains the number of unique improper torsion types in the
topology file. It contains one integer, the total number of improper torsions
types.

\begin{verbatim}
%FLAG CHARMM_NUM_IMPROPERS
%FORMAT(i8)
NIMPRTYPES
\end{verbatim}

\subsection*{CHARMM\_IMPROPER\_FORCE\_CONSTANT}

This section contains the force constant for each unique improper torsion type.
It is formatted exactly like {\tt DIHEDRAL\_FORCE\_CONSTANT}.

\noindent {\tt \%FORMAT(5E16.8)}

\noindent There are ${\tt NIMPRTYPES}$ integers in this section.

\subsection*{CHARMM\_IMPROPER\_PHASE}

This section contains the phase shift for each unique improper torsion type.
It is formatted exactly like {\tt DIHEDRAL\_PHASE}

\noindent {\tt \%FORMAT(5E16.8)}

\noindent There are ${\tt NIMPRTYPES}$ integers in this section.

\subsection*{LENNARD\_JONES\_14\_ACOEF}

Instead of scaling the 1-4 van der Waals interactions, the CHARMM force field
actually assigns entirely different LJ parameters to each atom type. Therefore,
\emph{chamber} topologies have two extra sections that correspond to the set of
LJ parameters for 1-4 interactions. The way these tables are set up is identical
to the way {\tt LENNARD\_JONES\_ACOEF} and {\tt LENNARD\_JONES\_BCOEF} are set
up in chamber topologies.

\noindent {\tt \%FORMAT(5E16.8)}

\noindent There are $\left [ {\tt NTYPES} \times ( {\tt NTYPES} + 1 ) \right ] /
2$ floating point numbers in this section.

\subsection*{LENNARD\_JONES\_14\_BCOEF}

\sloppy
This section contains the LJ B-coefficients for 1-4 interactions. See
{LENNARD\_JONES\_14\_ACOEF} above.

\noindent {\tt \%FORMAT(5E16.8)}

\noindent There are $\left [ {\tt NTYPES} \times ( {\tt NTYPES} + 1 ) \right ] /
2$ floating point numbers in this section.

\subsection*{CHARMM\_CMAP\_COUNT}

This section contains two integers---the number of total correction map (CMAP
terms and the number of unique CMAP `types.'

\begin{verbatim}
%FLAG CHARMM_CMAP_COUNT
%FORMAT(2i8)
CMAP_TERM_COUNT  CMAP_TYPE_COUNT
\end{verbatim}

\subsection*{CHARM\_CMAP\_RESOLUTION}

This section stores the resolution (\ie number of steps along each phi/psi CMAP
axis) for each CMAP grid.

\noindent {\tt \%FORMAT(20I4)}

\noindent There are {\tt CMAP\_TERM\_COUNT} integers in this section.

\subsection*{CHARMM\_CMAP\_PARAMETER\_#}

There are {\tt CMAP\_TYPE\_COUNT} of these sections, where # is replaced by a
2-digit integer beginning from 01. It is a 2-dimensional Fortran array whose 1-D
sequence is stored in column-major order.

\noindent {\tt \%FORMAT(8(F9.5))}

\noindent There are ${\tt CHARMM\_CMAP\_RESOLUTION(i)} ^ 2}$ floating point
numbers in this section, where $i$ is the # in the FLAG title.
